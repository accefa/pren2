\section{Kurzbeschreibung des kollaborativen Projektmanagements}
Der wesentliche Unterschied des kollaborativen Projektmanagements zu
klassischen Methoden ist jener, dass sämtliche Projektteilnehmer aktiv
in der Projektplanung und Projektsteuerung involviert sind. Dies
ermöglicht die Umsetzung einer weiteren Besonderheit des kollaborativen
Projektmanagements, nämlich jene der fachlichen Zuordnung und dezentralen
Projektsteuerung. Das heisst, dass die Gesamtplanung und Kontrolle des
Projekts auf Ebene der Fachbereiche basiert. Demzufolge gibt es eine starke
Kohärenz der Meilensteine, Arbeitspakete und deren Abwicklung. Für die
Koordination des Gesamtprojektes ist bei diesem Modell eine äusserst
starke Vernetzung notwendig. Diese muss gewährleisten, dass die Planung
und der Status des Projekts in genügender Tiefe jederzeit und für alle
Projektteilnehmer zugänglich ist.

Für die praktische Umsetzung dieses Modells gibt es eine Reihe von
besonderen Anforderungen, Arbeitsweisen und Strukturen. Einige Elemente
davon sind unter anderem:

\begin{itemize}
	\item Eine proaktive und Tool-gestützte Kollaboration.
	\item Eine zentrale Datenbasis.
	\item Klar zugeteilte und einsehbare Verantwortlichkeiten.
	\item Eine Top-Down-Vorgabe von Meilensteinen.
	\item Eine Intuitive Vernetzung aller Teilnehmer für Abstimmungen.
\end{itemize}
