\section{Arbeitsprozess}

Mit den beschriebenen Intrumenten für das Projektmanagment, welche durch
GitHub zur Verfügung gestellt werden und dem Modell des kollaborativen
Projektmanagements, hat sich ein Arbeitsprozess etabliert. Dieser Prozess,
wie in der Abbildung \ref{fig:process} dargestellt und ist im wesentlichen ein
Arbeitspaket-Regler.

\begin{figure}[h!]
	\centering
	\begin{tikzpicture}[node distance=2.85cm, scale=0.75, transform shape]
		\footnotesize
		\node (m) [flowchart-block] {define Milestone};
		\node (i) [flowchart-block, right of=m] {define Issue};
		\node (a) [flowchart-block, right of=i] {(re)assign Member};
		\node (ov) [flowchart-decision, below right of=a] {Assignee overload?};
		\node (si) [flowchart-block, right of=ov] {solve Issue};
		\node (is) [flowchart-decision, below of=si] {Issue solved?};
		\node (ci) [flowchart-block, right of=is] {close Issue};
		\node (np) [flowchart-decision, below of=i] {new problem?};
		\draw[blue, thick, ->] (m) -- (i);
		\draw[blue, thick, ->] (i) -- (a);
		\draw[blue, thick, ->] (a) -| (ov);
		\draw[blue, thick, ->] (ov) -| node[anchor=south, near start] {yes} (a);
		\draw[blue, thick, ->] (ov) -- node[anchor=south, near start] {no} (si);
		\draw[blue, thick, ->] (si) -- (is);
		\draw[blue, thick, ->] (is) -- node[anchor=south, near start] {yes} (ci);
		\draw[blue, thick, ->] (is) -| node[anchor=south, near start] {no} (np);
		\draw[blue, thick, ->] (np) -| node[anchor=south, near start] {no} (a);
		\draw[blue, thick, ->] (np) -- node[anchor=west, near start] {yes} (i);
	\end{tikzpicture}
	\caption{Etablierter Arbeitsprozess mit den Werkzeugen von GitHub}
	\label{fig:process}
\end{figure}

Als Startpunkt werden gemeinsam Meilensteine für das Projekt festgelegt. Dies
erfolgt in Anwesenheit und mit der Teilnahme aller Projektteilnehmer. Darauf
folgend werden zentrale Arbeitspakete definiert und intuitiv zugeteilt an
entsprechende Fachpersonen. In einem nächsten Schritt wird eine
Auslastungskontrolle durchgeführt, in welcher kontrolliert wird, ob eine
ungünstige Verteilung der Aufgaben vorliegt. Eine solche Unausgewogenheit
wird direkt an dieser Stelle versucht zu eliminieren. Ab diesem Zeitpunkt
beginnt die dezentrale Regelung. In dieser muss ein Teilnehmer selber
entscheiden, wie der weitere Verlauf des \gls{Issue} aussieht. Kann ein
\gls{Issue} nicht geschlossen werden, so muss der Verantwortliche dieses
neu Zuweisen (möglichst mit einem \gls{Comment} welcher diese Entscheidung
dokumentiert) oder ein neues \gls{Issue} definieren.

Dieser Prozess wird wöchentlich neu organisiert. D.h. der Status der
Meilsensteine wird wöchentlich kontrolliert und allfällige Probleme werden,
falls nicht schon vorher, mindestens im Wochentakt erkannt. 
