\section{Theorie der elementaren Intrumente des Projektmanagements}
Zu den elementaren Intrumenten des Projektmanagements gehören unter
anderen das Konzept der Meilensteine und Arbeitspakete. Im Folgenden
sollen diese beiden Konzepte, auf ihre wesentlichen Eigenschaften und
Bedeutungen hin, erläutert werden.

\subsection{Meilenstein}
Ein Meilsenstein stellt einen markanten Punkt dar für den
Projektfortschritt. Mit diesem Instrument ist es möglich ein Projekt
in mehrere Phasen zu unterteilen und deren Übergänge zu kontrollieren.
Somit sind Meilensteine ein Kontrollinstrument, welches den
Projektfluss steuert.

Für einen optimalen Einsatz dieses Intruments gibt es gewisse
Konventionen, deren Einhalten die Kontrolle des Projektflusses
begünstigen.

\begin{itemize}
	\item Meilensteine dienen der Übergangskontrolle von
		Projektphasen, smoit müssen diese linear und
		global auf das Projekt bezogen sein.
	\item Ein Meilenstein beschreibt Ziele, welche in konkreten
		Handlungen umsetzbar und messbar sind.
	\item Meilensteine haben einen zugeordneten Zeitpunkt, an
		welchen die Umsetzungkontrolle durchzuführen ist.
\end{itemize}

\subsection{Arbeitspaket}
Ein Arbeitspaket stellt im Kontext des Projektmanagements eine
Aufgabenstellung dar, welche sowohl zeitlich als auch inhaltlich
planbar und messbar ist. Somit stellt ein Arbeitspaket semantisch ein
Subset eines Meilensteins dar. Wie beim Meilenstein gibt es auch bei
den Arbeitspaketen einige Konventionen, welche den erfolgreichen
Einsatz begünstigen.

\begin{itemize}
	\item Ein Arbeitspaket ist identifizierbar.
	\item Ein Arbeitspaket hat eine Beschreibung, welche das Ziel
		und den Zweck nachvollziehbar darstellt.
	\item Ein Arbeitspaket hat eine klare Zuorndung zu einer
		Projektphase.
	\item Ein Arbeitspaket hat eine eindeutig zugewiesene
		Verantwortlichkeit.
	\item Ein Arbeitspaket hat klare Termininformationen.
\end{itemize}
