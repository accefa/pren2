\section{Issue}
Ein \gls{Issue} entpricht im Sinne des Projektmanagement einem Arbeitspaket.
Ein solches \gls{Issue} hat eine Reihe von Eigenschaften oder Attributen.

\subsection{Title}
Der \gls{Title} ist die Kurzbeschreibung des \gls{Issue}. Dieser ist kurz und
prägnant gehalten und gibt eine grobe Angabe über dessen Inhalt.

\subsection{Comment}
Der \gls{Comment} zum Arbeitspaket. Hierbei gibt es zwei Arten von
\gls{Comment} die es zu unterscheiden gilt:

\begin{itemize}
	\item Initialier \gls{Comment}
	\item Alle anderen \gls{Comment}
\end{itemize}

Der initiale \gls{Comment} stellt eine detaillierte Beschreibung des
\gls{Issue} dar. Alle anderen \gls{Comment} bilden ein chronologisch
geordnetes Forum.

\subsubsection{status \& event}
Ein \gls{Issue} hat einen definierten zeitlichen Startpunkt, welcher durch
seinen Instanzierungszeitpunkt gegeben ist. Eine existenzielle
Terminierung ist hingegen nicht vorgesehen. Der Lebenszyklus eines
\gls{Issue} ist nicht abgeschlossen, hat also nur einen definierten Anfang
und kein Ende seiner Lebenszeit.

\begin{figure}[h!]
	\centering
	\begin{tikzpicture}[node distance=2.85cm,scale=0.75,transform shape]
		\node (start) [flowchart-block] {issue is created};
		\node (open) [flowchart-block, right of=start] {issue is open};
		\node (closed) [flowchart-block, right of=open] {issue is closed};
		\draw[->, thick, blue] (start) -- (open);
		\draw[->, thick, blue] (open) -- (closed);
		\draw[->, thick, blue] (closed) -- ++(2,0) -| ++(0,-2) -| (open);
		\draw[->, thick, dotted, red] (closed) ++(2,0) -- ++(2,0) node[anchor=west] {issue at end of life};
	\end{tikzpicture}
	\caption{Lebenszyklus eines issue}
\end{figure}

Während dieses Lebenszyklus können sich verschiedene \gls{Event} ereignen.
Diese sind zwischen den \gls{Comment} dargestellt und somit ebenfalls
chronologisch geordnet. Mögliche \gls{Event} sind dabei

\begin{itemize}
	\item close (\emph{Abschluss})
	\item reopen (\emph{Wiedereröffnung})
	\item assignment (\emph{Änderung der Zuständigkeit})
\end{itemize}

\subsection{label}
Das \gls{Label} eines \gls{Issue} ist eine prägnante Markierung, welche mit
Text und Farbe eine Klassifizierung des \gls{Issue} darstellt. 

\subsection{milestone}
Ein issue kann zu einem milestone gelinkt werden. Im Sinne des
Projektmanagement bedeutet dies, dass das Arbeitspaket dem angegebenen
Meilenstein zugehörig ist. Somit ist der Abschluss des Arbeitspaket
eine zwingende Bedingung zur Erfüllen des Meilensteins.
