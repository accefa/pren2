\thispagestyle{empty}

\begin{abstract}
Die erfolgreiche Realisierung eines Projektes verlangt nach einer
gewissen Struktur und Organisation. Diese liegen insbesondere im
Bereich der Planung und kontinuierlichen Kontrolle. Als elementare
Intrumente gehören hierzu nach klassischen Modellen des
Projektmanagments die Meilensteine und Arbeitspakete.

In der vorliegenden Arbeit soll zunächst eine kurze Einführung in die
theoretischen Aspekte dieser Instrumente gegeben werden. In einem 
weiteren Abschnitt wird die Methode des kollaborativen
Projektmanagements soweit erläutert, wie es für die Verknüpfung für
die Verwendung mit GitHub notwendig ist. Die Beschreibung der
praktischen Umsetzung und der dazu eingesetzten Mittel stellt den
letzten Abschnitt der Arbeit dar, welcher auch einige Beispiele aus
der Umsetzung aufzeigt und Hinweise für die Verwendung gibt.

Ziel dieser Arbeit ist es, die Organisation und den Arbeitsprozess
hinsichtlich der Aspekte des Projektmanagements nachvollziehbar
aufzuzeigen, so dass eine Beurteilung der Projektplanung und des
Projektstatus durchgeführt werden kann.
\end{abstract}
