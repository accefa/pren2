\section{Management Summary}

Im Rahmen von PREN2 setzen wir das in PREN1 ausgewählte Lösungskonzept in die Realität um. Dadurch sammeln wir viel Erfahrung in der Zusammenarbeit mit verschiedenen Disziplinen. Dies tun ebenfalls einige andere Gruppen, mit denen wir in Konkurrenz stehen. Wir setzen uns zum Ziel, eine zuverlässige und robuste Maschine zu erschaffen, welche alle Bälle in den Korb befördert.
Zuerst definieren wir in der Gruppe die Schnittstellen zwischen Informatik, Elektrotechnik und Maschinenbau. Anschliessend arbeiten die einzelnen Disziplinen an ihren eigenen unabhängigen Bereichen.

Die Informatik befasst sich mit der Bedienung und der Logik des Aufbaus. Hierzu gehören ein GUI und Software für ein Raspberry Pi. Das Raspberry Pi liefert die Befehle verarbeitet an ein Freedom Board weiter. Diese Verbindung stellt die Schnittstelle zwischen Informatik und Elektrotechnik dar. Da die Maschine autonom funktionieren muss, wird vom Raspberry Pi ein WLAN Hotspot und ein Webserver gestartet. Somit ist es möglich, bequem aus der Distanz eine Verbindung aufzubauen und die Befehle über das GUI abzuschicken.

Die Teilgruppe Maschinenbau beschäftigt sich mit der Umsetzung der Konstruktion. Diese umfasst drei Motoren mit unterschiedlicher Funktion, welche jeweils speziell montiert sind. Das Gesamtkonzept sieht vor die Maschine zu Beginn auszurichten und die Bälle nacheinander über ein Schwungrad abzuschiessen. Um die Bälle nachzuschieben wird eine drehbare, gelagerte Gewindestange montiert an der ein Holzplättchen hängt. Diese drückt die Bälle kontinuierlich weiter. Ohne Lagerung der Stange würde diese verklemmen, dies kann zu Problemen führen, hauptsächlich jedoch zu Lärm. Der Ausgang ist verengt und sorgt dafür, dass die Bälle an das Schwungrad gedrückt werden. Der richtige Anpressdruck wird durch variable Führungshalter gewährleistet. Dieser ist ausschlaggebend für die Wurfweite, die Umdrehungsgeschwindigkeit kann vernachlässigt werden.

Die Elektronik hat die Aufgabe die Motoren der Maschinenbauer anzusteuern und die Befehle der Informatik auszuführen.
Für jeden Motor existiert ein eigenes Board. Angesprochen werden sie vom Freedomboard welches die zentrale Steuerung ist.
Das zuständige Teammitglied ist dauerhaft überlastet und arbeitet Überstunden. Trotzdem läuft alles reibungslos.

Zum Schluss kann die Gruppe auf einen funktionierenden Ballwerfer zurück schauen. Die Maschine trifft jedes mal zuverlässig den Korb. Dank des Dralls treffen auch die schrägen Schüsse den Korb präzise. Die ganze Konstruktion ist jedoch schwerer geworden als angenommen. Die Kosten hingegen sind moderat und bleiben im Rahmen des Budgets.

Das gelernte beschränkt sich nicht nur auf das Fachwissen. Die sozialen Fähigkeiten sind ebenfalls wichtig. Somit können die individuellen Stärken der Teammitglieder ideal genutzt werden.