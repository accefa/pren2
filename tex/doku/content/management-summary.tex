\section{Management Summary}

Im Rahmen von PREN2 setzen wir das in PREN1 ausgewählte Lösungskonzept in die Realität um. Dadurch sammeln wir viel Erfahrung in der Zusammenarbeit mit verschiedenen Disziplinen. Dies tun ebenfalls einige andere Gruppen, mit denen wir in Konkurrenz stehen. Wir setzen uns zum Ziel, eine zuverlässige und robuste Maschine zu erschaffen, welche alle Bälle in den Korb befördert.
Zuerst definieren wir in der Gruppe die Schnittstellen zwischen Informatik, Elektrotechnik und Maschinenbau. Anschliessend arbeiten die einzelnen Disziplinen an ihren eigenen unabhängigen Bereichen.

Die Informatik befasst sich mit der Bedienung und der Logik des Aufbaus. Hierzu gehören ein GUI und Software für ein Raspberry Pi. Das Raspberry Pi liefert die Befehle verarbeitet an ein Freedomboard weiter. Diese Verbindung stellt die Schnittstelle zwischen Informatik und Elektrotechnik dar. Da die Maschine autonom funktionieren muss, wird auf dem Raspberry Pi ein WLAN Hotspot und ein Webserver gestartet. Somit ist es möglich, bequem aus der Distanz eine Verbindung aufzubauen und die Befehle über das GUI abzuschicken.

Die Teilgruppe Maschinenbau beschäftigt sich mit der Umsetzung der Konstruktion. Diese umfasst drei Motoren mit unterschiedlicher Funktion, welche jeweils speziell montiert sind. Das Gesamtkonzept sieht vor die Maschine zu Beginn auszurichten und die Bälle nacheinander über ein Wurfrad abzuschiessen. Um die Bälle nachzuladen wird eine drehbare, gelagerte Trapezgewindespindel montiert an der ein Mitnehmer befestigt ist. Diese drückt die Bälle kontinuierlich weiter. Der Ausgang ist verengt und sorgt dafür, dass die Bälle an das Wurfrad gedrückt werden. Der richtige Anpressdruck wird durch eine verstellbare Gegendruckplatte gewährleistet. Diese ist ausschlaggebend für die Wurfweite. Die Drehzahl des Schwungrades hat einen kleineren Einfluss auf die Wurfdistanz.

Die Elektronik hat die Aufgabe die Motoren anzusteuern und die Befehle der Informatik auszuführen.
Für jeden Motor existiert ein eigenes Board. Angesprochen werden sie vom Freedomboard welches die zentrale Steuerung ist.

Zum Schluss kann die Gruppe auf einen funktionierenden Ballwerfer zurück schauen. Die Maschine trifft den Korb zuverlässig. Die ganze Konstruktion ist jedoch schwerer geworden als angenommen. Die Kosten hingegen sind moderat und bleiben im Rahmen des Budgets.

Das gelernte beschränkt sich nicht nur auf das Fachwissen. Die sozialen Fähigkeiten sind ebenfalls wichtig. Somit können die individuellen Stärken der Teammitglieder ideal genutzt werden.