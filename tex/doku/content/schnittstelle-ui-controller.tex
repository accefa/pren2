\subsection{REST Schnittstelle}

Die Schnittstelle von der externen Steuereinheit zum Raspberry Pi wird über eine REST\footnote{Representational State Transfer}-Schnittstelle realisiert. Es wurden folgende Ressourcen definiert:
\begin{itemize}
	\item camera
	\item image
	\item start
\end{itemize}
Jede Ressource lässt sich mit HTML-Methoden (GET, PUT, POST usw.) abrufen oder verändern. Die Schnittstelle lässt sich ohne Authentifizierung nutzen und arbeiten mit dem JSON-Datenformat. Nachfolgend werden die einzelnen Ressourcen näher beschrieben.

\subsubsection{GET camera}

Diese Anfrage liefert die aktuellen Einstellungen der Bilderkennung zurück.\\

\textbf{Parameter}


\begin{tabular}{l p{16cm}}
	\texttt{config} & Die Konfiguration der Bilderkennung \\
	\texttt{line\_y} & Die Pixellinie welche analysiert werden soll um den Korb zu detektieren. \\
	\texttt{line\_h} & Bei der Detektierung wird nicht nur eine Pixellinie untersucht. Mit \texttt{area} kann ein Band definiert werden in dem die Linien analysiert werden. \\
	\texttt{greyscale\_threshold} & Schwellenwert der Objekterkennung. \\
	\texttt{greyscale} & \texttt{true} wenn ein Graustufen-Bild erzeugt werden soll. \\
	\texttt{quality} & Ein Wert zwischen 0 und 100 für die Qualität des JPEG-Bildes. \\
	\texttt{crop\_x} & Breite des Feldes das auf beiden Seiten des Bildes abgeschnitten wird. \\
	\texttt{contrast} & Ein Wert zwischen -100 und 100 für den Kontrast der Kamera. \\
	\texttt{image} & Der URL zum Bild der Kamera. Bei jedem Aufruf wird ein neues Bild erstellt.
\end{tabular}\\

\textbf{Beispiel Request}

\texttt{GET} 
\texttt{http://<raspberrypi-ip>/camera}\\

\textbf{Beispiel Result}

\begin{lstlisting}[caption=GET camera Result, label=lst:camera, tabsize=2]
{
	"config": {
		"line_y": 0, 
		"line_h": 1, 
		"greyscale_threshold": 0,
		"greyscale": true, 
		"quality": 90, 
		"crop_x": 0, 
		"contrast": 10, 
	},
	"image": "http://<raspberrypi-ip>/image/basket.jpeg"
}
\end{lstlisting}

\subsubsection{PUT camera}

Diese Anfrage verändert die Einstellungen der Bilderkennung.\\

\textbf{Beispiel Request}

\texttt{PUT}
\texttt{http://<raspberrypi-ip>/camera} \\
Body siehe Listing \ref{lst:camera}\\

\textbf{Beispiel Result}

\texttt{HTTP/1.1 200 OK} \\
\texttt{Content-Type: text/json} \\

\noindent
Wenn kein valides JSON übergeben wurde: \\

\noindent
\texttt{HTTP/1.1 406 NOT ACCEPTABLE} \\
\texttt{Content-Type: text/json}

\subsubsection{GET image}

Diese Anfrage liefert das Bild der Kamera zurück. Bei jeder Anfrage wird ein Bild erstellt.\\

\textbf{Beispiel Request}

\texttt{GET} 
\texttt{http://<raspberrypi-ip>/image}\\

\textbf{Beispiel Result}

Bild im JPEG-Format

\subsubsection{PUT start}

Diese Anfrage startet den Vorgang und übergibt die Callback-Adresse vom Steuergerät, damit der Stopp-Befehl zurück gesendet werden kann.\\

\textbf{Parameter}

\begin{tabular}{l p{16cm}}
	\textbf{url} & Die IP-Adresse des Steuergerätes. Diese wird benötigt um den Stopp-Befehl zurückzusenden.
\end{tabular}\\

\textbf{Beispiel Request}

\texttt{PUT} 
\texttt{http://<raspberrypi-ip>/start}

\begin{lstlisting}[caption=PUT start Request, tabsize=2]
{
	"url": "http://<steuergerät-ip>"
}
\end{lstlisting}

\textbf{Beispiel Result}

\texttt{HTTP/1.1 200 OK} \\
\texttt{Content-Type: text/json} \\

\noindent
Wenn keine URL übergeben wurde: \\

\noindent
\texttt{HTTP/1.1 406 NOT ACCEPTABLE} \\
\texttt{Content-Type: text/json}