\subsubsection{REST Schnittstelle}
\label{sec:rest-schnittstelle}

Die Schnittstelle von der externen Steuereinheit zum Raspberry Pi wird über eine REST\footnote{Representational State Transfer}-Schnittstelle realisiert. Es wurden folgende Ressourcen definiert:
\begin{itemize}
	\item \texttt{start}
	\item \texttt{camera}
	\item \texttt{drive}
	\item \texttt{logger}
\end{itemize}
Jede Ressource lässt sich mit HTTP-Methoden (\texttt{GET}, \texttt{PUT}, \texttt{POST} usw.) abrufen oder verändern. Die Schnittstellen lassen sich ohne Authentifizierung nutzen und arbeiten mit dem JSON-Datenformat. Nachfolgend werden die einzelnen Ressourcen näher beschrieben.

\paragraph{\texttt{GET /camera}}

Diese Anfrage liefert die aktuellen Einstellungen der Bilderkennung zurück. Nachfolgenden sind die Parameter der Kamera dokumentiert.
\label{tab:parameter-bilderkennung}
\begin{table}[h!]
	\begin{tabular}{l p{16cm}}
		\texttt{config} & Die Konfiguration der Bilderkennung \\
		\texttt{line\_y} & Die Pixellinie welche analysiert werden soll um den Korb zu detektieren. \\
		\texttt{line\_h} & Bei der Detektierung wird nicht nur eine Pixellinie untersucht. Mit \texttt{area} kann ein Band definiert werden in dem die Linien analysiert werden. \\
		\texttt{greyscale\_threshold} & Schwellenwert der Objekterkennung. \\
		\texttt{greyscale} & \texttt{true} wenn ein Graustufen-Bild erzeugt werden soll. \\
		\texttt{quality} & Ein Wert zwischen 0 und 100 für die Qualität des JPEG-Bildes. \\
		\texttt{crop\_x} & Breite des Feldes das auf beiden Seiten des Bildes abgeschnitten wird. \\
		\texttt{contrast} & Ein Wert zwischen -100 und 100 für den Kontrast der Kamera. \\
	\end{tabular}
	
\end{table}

\begin{table}[h!]
	\centering
	\begin{tabular}{|l|l|}
		\hline Anfrage an Raspberry Pi & Antwort von Raspberry Pi \\ 
		\hline \texttt{GET /camera HTTP/1.1} & \texttt{200 OK} \\
		\texttt{Host: <raspberrypi-ip>} & \texttt{Content-Type: text/json} \\
		& \\
		& \verb|{| \\
		& \verb|"greyscaleThreshold": 9,| \\
		& \verb|"cropX": 295,| \\
		& \verb|"lineY": 800,| \\
		& \verb|"lineH": 1,| \\
		& \verb|"greyscale": false,| \\
		& \verb|"image": "/static/detected.jpg",| \\
		& \verb|"quality": 19,| \\
		& \verb|"contrast": 10| \\
		& \verb|}| \\
		\hline 
	\end{tabular} 
	\caption{\texttt{GET /camera}}
	\label{tab:get-camera}
\end{table}

\newpage

\paragraph{\texttt{PUT /camera}}

Diese Anfrage verändert die Einstellungen der Bilderkennung.

\begin{table}[h!]
	\centering
	\begin{tabular}{|l|l|}
		\hline Anfrage an Raspberry Pi & Antwort von Raspberry Pi \\ 
		\hline \texttt{PUT /camera HTTP/1.1} & \texttt{200 OK} \\
		\texttt{Host: <raspberrypi-ip>} & \texttt{Content-Type: text/plain} \\
		\texttt{Content-Type: text/json} & \\
		& \\
		\verb|{| & \verb|Im Fehlerfall:| \\
		\verb|"greyscaleThreshold": 9,| & \\
		\verb|"cropX": 295,| & \verb|406 NOT ACCEPTABLE| \\
		\verb|"lineY": 800,| & \verb|Content-Type: text/json| \\
		\verb|"lineH": 1,| & \\
		\verb|"greyscale": false,| & \\
		\verb|"image": "/static/detected.jpg",| & \\
		\verb|"quality": 19,| & \\
		\verb|"contrast": 10| & \\
		\verb|}| & \\
		\hline 
	\end{tabular} 
	\caption{\texttt{PUT /camera}}
	\label{tab:put-camera}
\end{table}

\paragraph{\texttt{POST /camera}}

Diese Anfrage erzeugt ein neues Bild und lässt die Detektierung laufen. Es sind keine Parameter notwendig.

\begin{table}[h!]
	\centering
	\begin{tabular}{|l|l|}
		\hline Anfrage an Raspberry Pi & Antwort von Raspberry Pi \\ 
		\hline \texttt{POST /camera HTTP/1.1} & \texttt{200 OK} \\
		\texttt{Host: <raspberrypi-ip>} & \texttt{Content-Type: text/plain} \\
		\hline 
	\end{tabular} 
	\caption{\texttt{POST /camera}}
	\label{tab:post-camera}
\end{table}

\paragraph{\texttt{GET /logger}}

Diese Anfrage liefert das Log-File des Server zurück.

\begin{table}[h!]
	\centering
	\begin{tabular}{|l|l|}
		\hline Anfrage an Raspberry Pi & Antwort von Raspberry Pi \\ 
		\hline \texttt{GET /logger HTTP/1.1} & \texttt{200 OK} \\
		\texttt{Host: <raspberrypi-ip>} & \texttt{Content-Type: text/plain} \\
		& \\
		& \verb|15.05.2015 15:16:33 INFO Starte Prozess| \\
		& \verb|15.05.2015 15:16:34 INFO Richte Stepper aus| \\
		& \verb|15.05.2015 15:16:39 INFO Starte BLDC| \\
		& \verb|15.05.2015 15:16:59 INFO Ballnachschub starten| \\
		& \verb|15.05.2015 15:18:02 INFO Prozess beendet| \\
		\hline 
	\end{tabular} 
	\caption{\texttt{GET /logger}}
	\label{tab:get-logger}
\end{table}

\newpage

\paragraph{\texttt{POST /drive/bldc}}

Für den BLDC gibt es drei Ressourcen:
\begin{itemize}
	\item \texttt{drive/bldc/start}
	\item \texttt{drive/bldc/stop}
	\item \texttt{drive/bldc/reset}
\end{itemize}
Für start braucht es die Drehzahl des Schwungrades in RPM\footnote{Rounds per Minute}. Die Drehzahl muss zwischen 4000 und 20000 liegen.

\begin{table}[h!]
	\centering
	\begin{tabular}{|l|l|}
		\hline Anfrage an Raspberry Pi & Antwort von Raspberry Pi \\ 
		\hline \texttt{POST /drive/bldc/start HTTP/1.1} & \texttt{200 OK} \\
		\texttt{Host: <raspberrypi-ip>} & \texttt{Content-Type: text/plain} \\
		\texttt{Content-Type: text/json} & \\
		& \\
		\verb|{| & \\
		\verb|"rpm": 5000| & \\
		\verb|}|& \\
		\hline 
	\end{tabular} 
	\caption{\texttt{POST /drive/bldc/start}}
	\label{tab:post-drive-bldc-start}
\end{table}

\begin{table}[h!]
	\centering
	\begin{tabular}{|l|l|}
		\hline Anfrage an Raspberry Pi & Antwort von Raspberry Pi \\ 
		\hline \texttt{POST /drive/bldc/stop HTTP/1.1} & \texttt{200 OK} \\
		\texttt{Host: <raspberrypi-ip>} & \texttt{Content-Type: text/plain} \\
		\hline 
	\end{tabular} 
	\caption{\texttt{POST /drive/bldc/stop}}
	\label{tab:post-drive-bldc-stop}
\end{table}

\begin{table}[h!]
	\centering
	\begin{tabular}{|l|l|}
		\hline Anfrage an Raspberry Pi & Antwort von Raspberry Pi \\ 
		\hline \texttt{POST /drive/bldc/reset HTTP/1.1} & \texttt{200 OK} \\
		\texttt{Host: <raspberrypi-ip>} & \texttt{Content-Type: text/plain} \\
		\hline 
	\end{tabular} 
	\caption{\texttt{POST /drive/bldc/reset}}
	\label{tab:post-drive-bldc-reset}
\end{table}

\paragraph{\texttt{POST /drive/dc}}

Für den DC gibt es vier Ressourcen: 
\begin{itemize}
	\item \texttt{drive/dc/forward}
	\item \texttt{drive/cd/backward}
	\item \texttt{drive/dc/stop}
	\item \texttt{drive/dc/reset}
\end{itemize}
Alle Anfragen verlangen keine Parameter und alle Antworten sind leer. Es wird nur die Anfrage \texttt{drive/dc/forward} dokumentiert. Die restlichen Anfragen sind gleich auszuführen.

\begin{table}[h!]
	\centering
	\begin{tabular}{|l|l|}
		\hline Anfrage an Raspberry Pi & Antwort von Raspberry Pi \\ 
		\hline \texttt{POST /drive/dc/forward HTTP/1.1} & \texttt{200 OK} \\
		\texttt{Host: <raspberrypi-ip>} & \texttt{Content-Type: text/plain} \\
		\hline 
	\end{tabular} 
	\caption{\texttt{POST /drive/dc/forward}}
	\label{tab:post-drive-dc-forward}
\end{table}

\paragraph{\texttt{POST /drive/stp}}

Für den STP gibt es zwei Ressourcen: 
\begin{itemize}
	\item \texttt{drive/stp/start}
	\item \texttt{drive/stp/reset}
\end{itemize}
Der Anfrage an die Ressource \texttt{drive/stp/start} muss die Anzahl der Schritte mitgeteilt werden, welche der Schrittmotor fahren muss. Die Anfrage an die Ressource \texttt{drive/stp/reset} hat keine Parameter. Die Antwort aller Anfragen ist leer.

\begin{table}[h!]
	\centering
	\begin{tabular}{|l|l|}
		\hline Anfrage an Raspberry Pi & Antwort von Raspberry Pi \\ 
		\hline \texttt{POST /drive/stp/start HTTP/1.1} & \texttt{200 OK} \\
		\texttt{Host: <raspberrypi-ip>} & \texttt{Content-Type: text/plain} \\
		\texttt{Content-Type: text/json} & \\
		& \\
		\verb|{| & \\
		\verb|"steps": 1000| & \\
		\verb|}|& \\
		\hline 
	\end{tabular} 
	\caption{\texttt{POST /drive/stp/start}}
	\label{tab:post-drive-stp-start}
\end{table}

\begin{table}[h!]
	\centering
	\begin{tabular}{|l|l|}
		\hline Anfrage an Raspberry Pi & Antwort von Raspberry Pi \\ 
		\hline \texttt{POST /drive/stp/reset HTTP/1.1} & \texttt{200 OK} \\
		\texttt{Host: <raspberrypi-ip>} & \texttt{Content-Type: text/plain} \\
		\hline 
	\end{tabular} 
	\caption{\texttt{POST /drive/stp/reset}}
	\label{tab:post-drive-stp-reset}
\end{table}

\paragraph{\texttt{PUT /start}}

Diese Anfrage startet den Wurf-Vorgang und ist blockierend. Es müssen keine Parameter übergeben werden.

\begin{table}[h!]
	\centering
	\begin{tabular}{|l|l|}
		\hline Anfrage an Raspberry Pi & Antwort von Raspberry Pi \\ 
		\hline \texttt{PUT /start HTTP/1.1} & \texttt{200 OK} \\
		\texttt{Host: <raspberrypi-ip>} & \texttt{Content-Type: text/plain} \\
		\hline 
	\end{tabular} 
	\caption{\texttt{PUT /start}}
	\label{tab:put-start}
\end{table}

