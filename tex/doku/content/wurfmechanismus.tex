\subsubsection{Wurfmechanismus}
\subsubsection*{Komponentenbeschrieb}

Hauptstück des Wurfmechanismus ist das Wurfrad, welches durch Drehung um seine eigene Achse die Bälle beschleunigt.
Das aus fünf MDF-Tellern bestehende Wurfrad ist auf einer Aluminium-Achse montiert. Die Achse ist mit zwei an seinen Enden angebrachten Radiallagern mit den Seitenwänden des Drehturmes verbunden. Auf der Achse befindet sich ein Zahnriemenrad, welches über einen Zahnriemen die Verbindung zum antreibenden BLDC-Motor ermöglicht.
Um die Reibung zwischen den Bällen und dem Wurfrad zu erhöhen wurde ein Gummiband auf das Wurfrad geklebt.


\subsubsection*{Entwicklungsprozess}

Da der Wurfmechanismus die zentrale Einheit der Maschine darstellt, wurde ein spezielles Augenmerk auf ihn gerichtet. Würde er nicht funktionieren, wäre die gesamte Konstruktion untauglich. Nach diversen mehr oder weniger sinnvollen Lösungsansätzen, entschied man sich daher für eine der konservativeren Bauarten.
Das Wurfrad als beschleunigendes Element hat sich bei vielen anderen Tennisball-Wurfmaschinen bewährt und bot sich daher als sichere Lösung an.