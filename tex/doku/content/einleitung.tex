\section{Einleitung}

Suchen, anvisieren und treffen. Zur Urzeiten waren diese Tätigkeiten überlebensnotwendig, wollte man etwas Fleisch zwischen die Zähne bekommen. Einige Zeitalter später steht man beinahe vor der gleichen Problemstellung. Jedoch soll kein Blut fliessen. Denn das Ziel ist kein lebendiges Tier sondern ein schwarzer, ruhiger Korb. Dieser Korb muss gefunden werden, die Ausrichtung muss diesen anvisieren und schlussendlich müssen fünf Tennisbälle hinein befördert werden.

In der Aufgabenstellung wurde eine Ballwurfmaschine gefordert, welche nach einem Startsignal, selbständig den Korb findet und die Bälle hinein befördert. Das PREN Team 39 hat in einer ersten Phase ein Konzept für eine solche Maschine erarbeitet. Es wurden verschiedene Lösungsansätze verfolgt und nach einem einheitlichen Raster bewertet. Die Idee mit der höchsten Punktzahl wurde im Detail ausgearbeitet um diese schlussendlich in der zweiten Phase umzusetzen. Angehende Maschinenbau-, Elektrotechnik- und Informatik-Ingenieure setzten sich zusammen um das Konzept optimal umzusetzen. Die Interdisziplinarität spielt eine grosse Rolle und ist ein wesentlicher Bestandteil des Projekts.

In der vorliegenden Arbeit wird zuerst grob auf das Lösungskonzept eingegangen. Anschliessend werden in den einzelnen Disziplinen die Komponenten beschrieben. Der mechanische Aufbau, die Verkabelung der Elektrotechnik sowie die Architektur der Software sind festgehalten. Informationen zum Projektmanagement wie die Organisation, die Planung, die Kosten und das Risikomanagement liegen nachfolgend vor. Detail-Informationen sind im Anhang vorhanden.


