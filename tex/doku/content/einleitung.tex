\section{Einleitung}

Automatisierte Systeme prägen die technische Entwicklung mit einer Dynamik,
wie dies einst die Dampfmaschine, der Verbrennungsmotor und nicht zuletzt der
Computer machten. Spezialisierte als auch alltägliche Aufgaben und 
Herausforderungen, die auch viel Geschick und Kraft benötigen, können immer
mehr durch autonome Maschinen durchgeführt werden. Dies liegt insbesondere
daran, dass moderne Technik nicht mehr nur der kaufkräftigen Industrie zur
Verfügung steht, sondern auch dem preiselastischen privaten Markt. Dass dies
heute möglich ist, hat verschiedene Gründe. Zum einen liegt dies an der
fortschreitenden technologischen Entwicklung. Ein wichtiger Katalysator für
die rasante Entwicklung ist der Paradigmenwechsel im Umgang mit Know-How.
Ein herausragendes Beispiel hierzu stellt die sogenannte 
``freie Software'' dar, welche mit berühmten Projekten wie GNU vieles
unserer heutigen technischen Wunder, wie etwa dem Internet, zu einer
Selbstverständlichkeit und Realität für Jedermann verhalf.
All diese Fortschritte führten zu einer immer mehr technisierten
Umgebung, welche die Zäune der Industrie längst überwunden hat und immer
mehr in den öffentlichen und privaten Raum Einzug hält. Dieser Wandel
verlangt nach intelligentem Design, einer optimalen Auslegung von
Ressourcen- und Energieverbrauch und nicht zuletzt nach einer
Kostenoptimierung. Gerade diese ermöglichen es dem einfachen Studenten an
moderne Technik zu gelangen und eine autonome Wurfmaschine zu
entwickeln. 

Die vorliegende Arbeit soll aufzeigen, wie die autonome Ballwurfmaschine aufgebaut und getestet wurde.
Im Fokus liegen die drei Motoren und der Abwurfmechanismus.
Wichtig hierbei sind die Test, die Resultate und die ergriffenen Massnahmen.
Zusätzlich wird die Organisation und die Arbeitsweise der Projektgruppe gezeigt,
sowie die Projektplanung und der tatsächliche Projektfortschritt.
