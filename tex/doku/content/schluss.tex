\section{Schluss}

Im PREN1 wurden Grobkonzepte und Lösungsvorschläge erarbeitet. Im PREN2 durfte das ausgewählte Konzept ''Drehrad'' umgesetzt werden. Das Team setzte sich zusammen und definierte wie die Umsetzung durchgeführt werden soll. Anschliessend konnte endlich mit der praktischen Arbeit gestartet werden. In Zusammenarbeit wurden die Disziplin übergreifenden Schnittstellen definiert. Wie gross ist das Raspberry-Pi? Wo darf es platziert werden? Wie viel Kraft benötigt der Schrittmotor? Wie schwer sind die Elektrotechnik Komponenten? All diese Fragen und noch viele weitere wurden während der Umsetzung gestellt und gelöst.

Nach wenigen Wochen konnten die ersten Teile zusammengesetzt werden. Wie zu erwarten, funktionierte nicht alles einwandfrei. Die gesamte Maschine konnte optimiert werden um den geforderten Erwartungen zu entsprechen. Zusätzliche Anforderungen für das Testing wurden erst während des Projektes Team-Intern definiert. So wollte man beispielsweise, dass die einzelnen Motoren per Knopf-Druck auf dem Notebook gestartet und gestoppt werden können. So entstand teilweise in der Informatik zusätzlicher Aufwand, welcher aber zu keinen zeitlichen Verzögerungen führte.

Durch seriöses Projektmanagement konnten alle Korrekturen überwacht und abgesprochen werden. So konnte frühzeitig reagiert werden, falls ein Ziel nicht oder verspätet erreicht wurde. Das Projekt konnte durch die engagierte Mitarbeit aller Beteiligten erfolgreich abgeschlossen werden.\\

\noindent
\textbf{Lessons learned}
\\
Das ganze Team konnte im letzten Jahr wertvolle Erfahrungen sammeln. Nachfolgend einige Punkte, welche wir besonders hervorheben möchten:

\begin{enumerate}
	\item  In einem interdisziplinären Projekt steht die \textbf{Zusammenarbeit} an oberster Stelle. Arbeitet eine Disziplin im Elfenbeinturm, dann wird ein erfolgreicher Projektabschluss schwierig.
	\item Schnittstellen-Definitionen sollten möglichst früh definiert werden, damit jeder weiss, was für Anforderungen die Komponente zu erfüllen hat. Auch benötigte Änderungen dieser Definitionen müssen schnell und klar \textbf{kommuniziert} werden.
	\item Möglichst schnell in die Komponenten-\textbf{Tests} einsteigen. Umso früher integrierte Komponenten getestet werden können, umso früher wird erkannt, was funktionieren kann und was nicht. Uns wurde so relativ schnell klar, dass die Konstruktion für die Ballführung verbessert werden muss, damit diese nicht blockiert.
	\item \textbf{Risiken} können sich ändern. Wurde zu Beginn die Ortung des Korbes als grosses Risiko wahrgenommen, weil Know-How in diesem Bereich vermisst wurde. Zeichnete sich ab, dass der Wurf-Mechanismus einige Risiken mehr mit sich brachte.
\end{enumerate}

\noindent
\textbf{Würdigung}
\\
Das PREN Team 39 bedankt sich bei allen Personen, welche einen Teil zum Projekt beigetragen haben. Ein besonderer Dank geht an den Projektbetreuer Herr \textbf{Martin Vogel}, welcher durch energisches Hinterfragen unserer Lösung einen wichtigen Teil zum Erfolg unseres Projektes beigetragen hat. Ein Teammitglied ist besonders hervorzuheben. Die Elektrotechnik war mit nur einem Mann schwach besetzt, hatte jedoch mit der Ansteuerung von drei Motoren enorm viel Aufwand. Die restlichen Teammitglieder bedanken sich bei \textbf{Ervin} für seinen unermüdlichen Einsatz.