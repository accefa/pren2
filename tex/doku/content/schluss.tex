\section{Schluss}

Im PREN1 wurden Grobkonzepte und Lösungsvorschläge erarbeitet. Nun konnte das ausgewählte Konzept ''Drehrad'' umgesetzt werden. Jedes Gruppenmitglied wusste was zu tun war und machte sich an die Arbeit.
Die Informatiker beschäftigten sich mit der Umsetzung des GUI's und dem Raspberry Pi Script.
In Zusammenarbeit mit der Elektrotechnik wurden Schnittstellen definiert und implementiert.
So kann die Informatik direkt vom Programm die Motoren ansteuern. Die Befehle werden auf dem Freedom Board geparst.
Die Einzelteile der kompletten Maschine wurden am Computer designed und in Auftrag gegeben.

Nach wenigen Wochen schon, konnten die ersten Teile zusammengesetzt werden. Wie zu erwarten, funktionierte nicht alles einwandfrei. Die Maschine konnte optimiert und die Boards neu konfiguriert werden um den Erwartungen zu entsprechen.
Selbstverständlich traf auch die Informatik auf Probleme und musste sich den neuen Gegebenheiten anpassen.
Zusätzlich wurde auch in gemischten Disziplinen gearbeitet. So konnten komplexe Aufgaben realisiert werden, wie z.B. die Ausrichtung der Maschine mithilfe des Kamerabilds oder das Troubleshooting verschwundener Befehle zwischen Raspberry und Freedom Board. Die Kalibrierung des Abwurfmechanismus, welcher für die Genauigkeit entscheidend ist, wurde jeweils von allen Mitgliedern gespannt verfolgt.

Durch seriöses Projektmanagement konnten alle Korrekturen überwacht und abgesprochen werden.
So konnte frühzeitig reagiert werden, falls ein Ziel nicht oder verspätet erreicht wurde.
Das Projekt konnte durch die engagierte Mitarbeit aller Beteiligten erfolgreich abgeschlossen werden.
