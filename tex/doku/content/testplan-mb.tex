\subsection{Testplan-Maschinenbau}
\subsubsection{Funktionskontrolle Schrittmotor}
\begin{table}[h!]
	\renewcommand{\arraystretch}{1.5}
	\begin{tabular}{|r|p{14cm}|}
		\hline Beschreibung & Schrittmotor muss den Aufbau in beide Drehrichtungen über den Schwenkbereich drehen können  \\ 
		\hline Vorbedingungen & Aufbau montiert \\ 
		\hline Vorgehen & 
		\begin{enumerate}
			\item Schrittmotor nach links drehen lassen 
			\item Schrittmotor nach rechts drehen lassen
		\end{enumerate} \\ 
		\hline Ergebnis & Funktion IO/ NIO \\ 
		\hline Verbesserungen & Reibung verringern, Lagerung anpassen, Positionierung Schrittmotor anpassen \\ 
		\hline 
	\end{tabular}
\end{table}

\subsubsection{Funktionskontrolle DC- Motor}
\begin{table}[h!]
	\renewcommand{\arraystretch}{1.5}
	\begin{tabular}{|r|p{14cm}|}
		\hline Beschreibung & DC-Motor muss die Bälle dem Drehrad zuführen können.  \\ 
		\hline Vorbedingungen & Aufbau montiert \\ 
		\hline Vorgehen & 
		\begin{enumerate}
			\item Motor in beide Drehrichtungen drehen lassen (ohne Bälle). 
			\item Test mit Bällen wiederholen
		\end{enumerate} \\ 
		\hline Ergebnis & Funktion IO/ NIO \\ 
		\hline Verbesserungen & Konstruktive Anpassungen, Drezahl des Motors anpassen \\ 
		\hline 
	\end{tabular}
\end{table}
\newpage

\subsubsection{Funktionskontrolle BLDC- Motor}
\begin{table}[h!]
	\renewcommand{\arraystretch}{1.5}
	\begin{tabular}{|r|p{14cm}|}
		\hline Beschreibung & BLDC-Motor muss die Bälle auf die gewünschte Abwurfgeschwindigkeit beschleunigen.  \\ 
		\hline Vorbedingungen & Aufbau montiert \\ 
		\hline Vorgehen & 
		\begin{enumerate}
			\item BLDC- Motor auf gewünschter Drezahl Drehen lassen 
			\item Mechanischer Aufbau überprüfen
			\item Test mit Bällen wiederholen 
		\end{enumerate} \\ 
		\hline Ergebnis & Funktion IO/ NIO \\ 
		\hline Verbesserungen & Konstruktive Anpassungen, Drezahl des Motors anpassen, Gegendruckplatte konstruktiv anpassen \\ 
		\hline 
	\end{tabular}
\end{table}

\subsubsection{Stabilität Aufbau}
\begin{table}[h!]
	\renewcommand{\arraystretch}{1.5}
	\begin{tabular}{|r|p{14cm}|}
		\hline Beschreibung & Der Aufbau muss bei maximaler Drezahl aller Motoren stabil betrieben werden können. \\ 
		\hline Vorbedingungen & Funktionskontrolle der einzelnen Motoren gewährleistet. \\ 
		\hline Vorgehen & 
		\begin{enumerate}
			\item BLDC- Motor auf maximale Drezahl bringen
			\item 2 min drehen lassen 
			\item Schrittmotor mit max. Geschwindigkeit über den ganzen Schwenkbereich drehen lassen  
			\item DC- Motor einschalten
			\item Kontrolle des mechanischen Aufbaus
		\end{enumerate} \\ 
		\hline Ergebnis & Keine Schäden am Aufbau erkennbar/ Schäden am Aufbau erkennbar \\ 
		\hline Verbesserungen & Stabilität anpassen, Drehzahlbereich der Motoren einschränken \\ 
		\hline 
	\end{tabular}
\end{table}
\newpage

\subsubsection{Funktionskontrolle Treffgenauigkeit }
\begin{table}[h!]
	\renewcommand{\arraystretch}{1.5}
	\begin{tabular}{|r|p{14cm}|}
		\hline Beschreibung & Die Bälle müssen in der gewünschten Zeit die gewünschte Treffgenauigkeit erreichen   \\ 
		\hline Vorbedingungen & Einzelfunktionskontrollen aller Motoren erfolgreich, Stabilität des Aufbaus erreicht \\ 
		\hline Vorgehen & 
		\begin{enumerate}
			\item BLDC- Motor auf gewünschter Drehzahl Drehen lassen 
			\item Ballnachschub starten 
		\end{enumerate} \\ 
		\hline Ergebnis & Funktion IO/ NIO \\ 
		\hline Verbesserungen & Konstruktive Anpassungen, Drezahl des Motors anpassen, Gegendruckplatte konstruktiv anpassen \\ 
		\hline 
	\end{tabular}
\end{table}