\section{Testplan}
Es wird zwischen Integrations- (INT) und Systemtest (SYS) unterschieden. Ein Integrationstest
ist technisch getrieben und prüft das Zusammenspiel der Komponenten. Aus diesem
Grund wird empfohlen, dass solche Tests von den Entwicklern selber durchgeführt werden.
Systemtest hingegen sind benutzerorientiert. Diese testen das System als Ganzes.
Die Tests sind voneinander abhängig, darum ist die Reihenfolge der Testausführung
definiert. Denn ein Test kann als Vorbedingung für einen anderen Test gelten. 

In den Tests wird von Controller und Configurator gesprochen. Der Controller ist
das Raspberry Pi und der Configurator die externe Steuerungseinheit.

\subsection{Reihenfolge der Testdurchführung}
INT101, INT102, INT103
SYS101, SYS102, SYS103

\subsection{INT101 Webservice Erreichbarkeit}
\begin{table}[h!]
	\renewcommand{\arraystretch}{1.5}
	\begin{tabular}{|r|p{14cm}|}
		\hline Beschreibung & Die Webservices auf dem Controller müssen erreichbar sein. \\ 
		\hline Vorbedingungen & Controller Setup \\ 
		\hline Testdaten & URL: http://<IP_ADDRESS_CONTROLLER>:<PORT_CONTROLLER> \\ 
		\hline Vorgehen & 
		\begin{enumerate}
			\item Browser öffnen und URL eingeben.
		\end{enumerate} \\ 
		\hline Ergebnis & Der Browser zeigt eine Willkommensseite an \\ 
		\hline 
	\end{tabular}
\end{table}

\subsection{INT102 Webservice Bild laden}
\begin{table}[h!]
	\renewcommand{\arraystretch}{1.5}
	\begin{tabular}{|r|p{14cm}|}
		\hline Beschreibung & Der Bild-Lade Webservice muss ein Bild zurückliefern. \\ 
		\hline Vorbedingungen & INT101 \\ 
		\hline Testdaten & URL: http://<IP_ADDRESS_CONTROLLER>:<PORT_CONTROLLER>/image \\ 
		\hline Vorgehen & 
		\begin{enumerate}
			\item Browser öffnen und URL eingeben
		\end{enumerate} \\ 
		\hline Ergebnis & Ein Bild wird angezeigt. \\ 
		\hline 
	\end{tabular}
\end{table}

\subsection{INT103 Webservice Kamera Konfig laden}
\begin{table}[h!]
	\renewcommand{\arraystretch}{1.5}
	\begin{tabular}{|r|p{14cm}|}
		\hline Beschreibung & Der Kamera-Konfig-Lade Webservice muss ein JSON-File mit den aktuellen Einstellungen liefern. \\ 
		\hline Vorbedingungen & INT102 \\ 
		\hline Testdaten & URL: http://<IP_ADDRESS_CONTROLLER>:<PORT_CONTROLLER>/camera \\ 
		\hline Vorgehen & 
		\begin{enumerate}
			\item Browser öffnen und URL eingeben
		\end{enumerate} \\ 
		\hline Ergebnis & JSON-File mit aktuellen Einstellungen ist ersichtlich. \\ 
		\hline 
	\end{tabular}
\end{table}

\subsection{SYS101 Configurator startet}
\begin{table}[h!]
	\renewcommand{\arraystretch}{1.5}
	\begin{tabular}{|r|p{14cm}|}
		\hline Beschreibung & Der Configurator lässt sich starten. \\ 
		\hline Vorbedingungen & Configurator Setup \\ 
		\hline Testdaten &  \\ 
		\hline Vorgehen & 
		\begin{enumerate}
			\item Configurator starten
		\end{enumerate} \\ 
		\hline Ergebnis & Es erscheint eine Benutzeroberfläche. \\ 
		\hline 
	\end{tabular}
\end{table}

\subsection{SYS102 Configurator Verbindungseinstellungen }
\begin{table}[h!]
	\renewcommand{\arraystretch}{1.5}
	\begin{tabular}{|r|p{14cm}|}
		\hline Beschreibung & Im Configurator lassen sich die Verbindungseinstellungen zum Controller ändern. \\ 
		\hline Vorbedingungen & SYS101 \\ 
		\hline Testdaten & Eigene Verbindungsdaten definieren \\ 
		\hline Vorgehen & 
		\begin{enumerate}
			\item Configurator starten
			\item Verbindungeinstellungen ändern und speichern
			\item Configurator neu starten
		\end{enumerate} \\ 
		\hline Ergebnis & Verbindungseinstellungen sind auch nach dem Neustart erhalten. \\ 
		\hline 
	\end{tabular}
\end{table}

\subsection{SYS103 Configurator Bild laden }
\begin{table}[h!]
	\renewcommand{\arraystretch}{1.5}
	\begin{tabular}{|r|p{14cm}|}
		\hline Beschreibung & Aktuelle Foto lässt sich laden und im Configurator anzeigen. \\ 
		\hline Vorbedingungen & SYS102 \\ 
		\hline Testdaten & Verbindungsdaten definieren, Bild \\ 
		\hline Vorgehen & 
		\begin{enumerate}
			\item Configurator starten
			\item Bild laden
		\end{enumerate} \\ 
		\hline Ergebnis & Aktuelles Foto erscheint im GUI. \\ 
		\hline 
	\end{tabular}
\end{table}

\subsection{SYS104 Configurator Konfiguration anwenden }
\begin{table}[h!]
	\renewcommand{\arraystretch}{1.5}
	\begin{tabular}{|r|p{14cm}|}
		\hline Beschreibung & Die Kamera-Einstellungen lassen sich editieren und auf den Controller anwenden. \\ 
		\hline Vorbedingungen & SYS103 \\ 
		\hline Testdaten & Konfigurationsdaten \\ 
		\hline Vorgehen & 
		\begin{enumerate}
			\item Configurator starten
			\item Kamera-Einstellunge editieren
			\item Kamera-Einstellung speichern
		\end{enumerate} \\ 
		\hline Ergebnis & Auf dem GUI erscheint die Meldung, dass die Konfiguration erfolgreich gespeichert wurde. \\ 
		\hline 
	\end{tabular}
\end{table}