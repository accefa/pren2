\section{Testplan}

Der Testplan beschreibt alle wichtigen Testfälle und ihre erwarteten Resultate.

\subsection{Ablaufplan}
\begin{figure}[h!]
	\centering
	\includegraphics[width=0.7\linewidth]{../../fig/ablauf-ballwurf}
	\caption{}
	\label{fig:ablauf-ballwurf}
\end{figure}

\subsection{Systemtest}

\subsection{Testplan-Maschinebau}
\subsubsection{Funktionskontrolle Schrittmotor}
\begin{table}[h!]
	\renewcommand{\arraystretch}{1.5}
	\begin{tabular}{|r|p{14cm}|}
		\hline Beschreibung & Schrittmotor muss den Aufbau in beide Drehrichtungen über den Schwenkbereich drehen können, mit der gewünschten Genauigkeit  \\  
		\hline Vorbedingungen & Aufbau montiert \\ 
		\hline Vorgehen & 
		\begin{enumerate}
			\item Schrittmotor nach links drehen lassen 
			\item Schrittmotor nach rechts drehen lassen
		\end{enumerate} \\ 
		\hline Ergebnis & Funktion IO/ NIO \\ 
		\hline Verbesserungen & Reibung verringern, Lagerung anpassen, Positionierung Schrittmotor anpassen \\ 
		\hline 
	\end{tabular}
\end{table}

\subsubsection{Funktionskontrolle DC- Motor}
\begin{table}[h!]
	\renewcommand{\arraystretch}{1.5}
	\begin{tabular}{|r|p{14cm}|}
		\hline Beschreibung & DC-Motor muss die Bälle dem Drehrad zuführen können.  \\ 
		\hline Vorbedingungen & Aufbau montiert \\ 
		\hline Vorgehen & 
		\begin{enumerate}
			\item Motor in beide Drehrichtungen drehen lassen (ohne Bälle). 
			\item Test mit Bällen wiederholen
		\end{enumerate} \\ 
		\hline Ergebnis & Funktion IO/ NIO \\ 
		\hline Verbesserungen & Konstruktive Anpassungen, Drezahl des Motors anpassen \\ 
		\hline 
	\end{tabular}
\end{table}

\subsubsection{Funktionskontrolle BLDC- Motor}
\begin{table}[h!]
	\renewcommand{\arraystretch}{1.5}
	\begin{tabular}{|r|p{14cm}|}
		\hline Beschreibung & BLDC-Motor muss die Bälle auf die gewünschte Abwurfgeschwindigkeit und in die geforderte Richtung beschleunigen.  \\ 
		\hline Vorbedingungen & Aufbau montiert \\ 
		\hline Vorgehen & 
		\begin{enumerate}
			\item BLDC- Motor auf gewünschter Drezahl Drehen lassen 
			\item Mechanischer Aufbau überprüfen
			\item Test mit Bällen wiederholen 
		\end{enumerate} \\ 
		\hline Ergebnis & Funktion IO/ NIO \\ 
		\hline Verbesserungen & Konstruktive Anpassungen, Drezahl des Motors anpassen, Gegendruckplatte konstruktiv anpassen \\ 
		\hline 
	\end{tabular}
\end{table}

\subsubsection{Stabilität Aufbau}
\begin{table}[h!]
	\renewcommand{\arraystretch}{1.5}
	\begin{tabular}{|r|p{14cm}|}
		\hline Beschreibung & Der Aufbau muss bei maximaler Drezahl aller Motoren stabil betrieben werden können. \\ 
		\hline Vorbedingungen & Funktionskontrolle der einzelnen Motoren gewährleistet. \\ 
		\hline Vorgehen & 
		\begin{enumerate}
			\item BLDC- Motor auf maximale Drezahl bringen
			\item 2 min drehen lassen 
			\item Schrittmotor mit max. Geschwindigkeit über den ganzen Schwenkbereich drehen lassen  
			\item DC- Motor einschalten
			\item Kontrolle des mechanischen Aufbaus
		\end{enumerate} \\ 
		\hline Ergebnis & Keine Schäden am Aufbau erkennbar/ Schäden am Aufbau erkennbar \\ 
		\hline Verbesserungen & Stabilität anpassen, Drehzahlbereich der Motoren einschränken \\ 
		\hline 
	\end{tabular}
\end{table}

\subsubsection{Funktionskontrolle Treffgenauigkeit }
\begin{table}[h!]
	\renewcommand{\arraystretch}{1.5}
	\begin{tabular}{|r|p{14cm}|}
		\hline Beschreibung & Die Bälle müssen in der gewünschten Zeit die gewünschte Treffgenauigkeit erreichen   \\ 
		\hline Vorbedingungen & Einzelfunktionskontrollen aller Motoren erfolgreich, Stabilität des Aufbaus erreicht \\ 
		\hline Vorgehen & 
		\begin{enumerate}
			\item BLDC- Motor auf gewünschter Drehzahl Drehen lassen 
			\item Ballnachschub starten 
		\end{enumerate} \\ 
		\hline Ergebnis & Funktion IO/ NIO \\ 
		\hline Verbesserungen & Konstruktive Anpassungen, Drezahl des Motors anpassen, Gegendruckplatte konstruktiv anpassen \\ 
		\hline 
	\end{tabular}
\end{table}

\subsection{Testplan-Elektrotechnik}

\subsection{Gesamtsystemtest}

Im Gesamtsystemtest wird der Ballwerfer als Ganzes getestet. Im speziellen wird der Ablauf bei einem Probewurf simuliert.

\subsubsection{Reihenfolge der Testdurchführung}
GST101, GST102, GST103, GST104, GST105, GST106, GST107, GST108

\subsubsection{GST101 Ballwerfer ausgerichtet}
\begin{table}[h!]
	\renewcommand{\arraystretch}{1.5}
	\begin{tabular}{|r|p{14cm}|}
		\hline Beschreibung & Ballwerfer ist mit der Schablone auf dem Spielfeld mittig ausgerichtet. \\ 
		\hline Vorbedingungen &  \\ 
		\hline Testdaten & Konfigurationsdaten \\ 
		\hline Vorgehen & 
		\begin{enumerate}
			\item Schablone auflegen
			\item Ballwerfer auf die Mitte stellen
			\item Schablone entfernen
		\end{enumerate} \\ 
		\hline Ergebnis & Der Ballwerfer steht in der Mitte des Spielfeld.s \\ 
		\hline 
	\end{tabular}
\end{table}

\subsubsection{GST102 Strom angeschlossen}
\begin{table}[h!]
	\renewcommand{\arraystretch}{1.5}
	\begin{tabular}{|r|p{14cm}|}
		\hline Beschreibung & Alle Komponenten sind am Strom angeschlossen. \\ 
		\hline Vorbedingungen &  GST101\\ 
		\hline Testdaten & Konfigurationsdaten \\ 
		\hline Vorgehen & 
		\begin{enumerate}
			\item Netzteil am Strom anschliessen
			\item Rasperry am Netzteil anschliessen
			\item Freedom Board am Netzteil anschliessen
			\item DC Motor am Netzteil anschliessen
			\item BLDC Motor am Netzteil anschliessen
			\item STP Motor am Netzteil anschliessen
		\end{enumerate} \\ 
		\hline Ergebnis & Alle Komponenten sind am Strom angeschlossen und lassen sich in Betrieb nehmen. \\ 
		\hline 
	\end{tabular}
\end{table}
\newpage

\subsubsection{GST103 Boards eingeschaltet}
\begin{table}[h!]
	\renewcommand{\arraystretch}{1.5}
	\begin{tabular}{|r|p{14cm}|}
		\hline Beschreibung & Freedom Board und Rasperry einschalten. \\ 
		\hline Vorbedingungen &  GST102\\ 
		\hline Testdaten & Konfigurationsdaten \\ 
		\hline Vorgehen & 
		\begin{enumerate}
			\item Freedom Board einschalten
			\item Rasperry einschalten
		\end{enumerate} \\ 
		\hline Ergebnis & Freedom Board und Rasperry sind in Betrieb. \\ 
		\hline 
	\end{tabular}
\end{table}

\subsubsection{GST104 Verbindung aufgebaut}
\begin{table}[h!]
	\renewcommand{\arraystretch}{1.5}
	\begin{tabular}{|r|p{14cm}|}
		\hline Beschreibung & Computer ist mit dem Rasperry WLAN verbunden. \\ 
		\hline Vorbedingungen &  GST103\\ 
		\hline Testdaten & Konfigurationsdaten \\ 
		\hline Vorgehen & 
		\begin{enumerate}
			\item Computer einschalten
			\item WLAN verbinden
			\item WLAN Login
		\end{enumerate} \\ 
		\hline Ergebnis & Computer ist mit dem Rasperry WLAN verbunden. \\ 
		\hline 
	\end{tabular}
\end{table}

\subsubsection{GST105 GUI geöffnet}
\begin{table}[h!]
	\renewcommand{\arraystretch}{1.5}
	\begin{tabular}{|r|p{14cm}|}
		\hline Beschreibung & Der Configurator lässt sich auf dem Computer öffnen. \\ 
		\hline Vorbedingungen &  GST104\\ 
		\hline Testdaten & Konfigurationsdaten \\ 
		\hline Vorgehen & 
		\begin{enumerate}
			\item Configurator starten
		\end{enumerate} \\ 
		\hline Ergebnis & Der Configurator ist geöffnet und lässt sich bedienen. \\ 
		\hline 
	\end{tabular}
\end{table}
\newpage

\subsubsection{GST106 Reset Motoren}
\begin{table}[h!]
	\renewcommand{\arraystretch}{1.5}
	\begin{tabular}{|r|p{14cm}|}
		\hline Beschreibung & Alle Motoren haben einen Reset durchgeführt. \\ 
		\hline Vorbedingungen &  GST105\\ 
		\hline Testdaten & Konfigurationsdaten \\ 
		\hline Vorgehen & 
		\begin{enumerate}
			\item GUI Tab wechseln
			\item STP Motor reseten
			\item DC Motor reseten
			\item BLDC Motor reseten
		\end{enumerate} \\ 
		\hline Ergebnis & Alle Motoren haben ihren Ausgangsstatus. \\ 
		\hline 
	\end{tabular}
\end{table}

\subsubsection{GST107 Bälle eingefüllt}
\begin{table}[h!]
	\renewcommand{\arraystretch}{1.5}
	\begin{tabular}{|r|p{14cm}|}
		\hline Beschreibung & Die Bälle zum Abschuss einfüllen. \\ 
		\hline Vorbedingungen &  GST106\\ 
		\hline Testdaten & Konfigurationsdaten \\ 
		\hline Vorgehen & 
		\begin{enumerate}
			\item 5 Bälle einfüllen
		\end{enumerate} \\ 
		\hline Ergebnis & Der Ballwerfer ist geladen. \\ 
		\hline 
	\end{tabular}
\end{table}

\subsubsection{GST108 Kamera kalibriert}
\begin{table}[h!]
	\renewcommand{\arraystretch}{1.5}
	\begin{tabular}{|r|p{14cm}|}
		\hline Beschreibung & Kamera kalibrieren \\ 
		\hline Vorbedingungen &  GST104\\ 
		\hline Testdaten & Konfigurationsdaten \\ 
		\hline Vorgehen & 
		\begin{enumerate}
			\item GUI Tab wechseln
			\item Werte im Configurator einstellen
		\end{enumerate} \\ 
		\hline Ergebnis & Die Kamera ist so eingestellt, dass sie den Korb erkennt. \\ 
		\hline 
	\end{tabular}
\end{table}

\newpage
\subsection{Testplan-Maschinenbau}
\subsubsection{MB101 Funktionskontrolle Schrittmotor}
\begin{table}[h!]
	\renewcommand{\arraystretch}{1.5}
	\begin{tabular}{|r|p{14cm}|}
		\hline Beschreibung & Schrittmotor muss den Aufbau in beide Drehrichtungen über den Schwenkbereich drehen können  \\ 
		\hline Vorbedingungen & Aufbau montiert \\ 
		\hline Vorgehen & 
		\begin{enumerate}
			\item Schrittmotor nach links drehen lassen 
			\item Schrittmotor nach rechts drehen lassen
		\end{enumerate} \\ 
		\hline Ergebnis & Funktion IO/ NIO \\ 
		\hline Verbesserungen & Reibung verringern, Lagerung anpassen, Positionierung Schrittmotor anpassen \\ 
		\hline 
	\end{tabular}
\end{table}

\subsubsection{MB102 Funktionskontrolle DC- Motor}
\begin{table}[h!]
	\renewcommand{\arraystretch}{1.5}
	\begin{tabular}{|r|p{14cm}|}
		\hline Beschreibung & DC-Motor muss die Bälle dem Drehrad zuführen können.  \\ 
		\hline Vorbedingungen & Aufbau montiert \\ 
		\hline Vorgehen & 
		\begin{enumerate}
			\item Motor in beide Drehrichtungen drehen lassen (ohne Bälle). 
			\item Test mit Bällen wiederholen
		\end{enumerate} \\ 
		\hline Ergebnis & Funktion IO/ NIO \\ 
		\hline Verbesserungen & Konstruktive Anpassungen, Drezahl des Motors anpassen \\ 
		\hline 
	\end{tabular}
\end{table}
\newpage

\subsubsection{MB103 Funktionskontrolle BLDC- Motor}
\begin{table}[h!]
	\renewcommand{\arraystretch}{1.5}
	\begin{tabular}{|r|p{14cm}|}
		\hline Beschreibung & BLDC-Motor muss die Bälle auf die gewünschte Abwurfgeschwindigkeit beschleunigen.  \\ 
		\hline Vorbedingungen & Aufbau montiert \\ 
		\hline Vorgehen & 
		\begin{enumerate}
			\item BLDC- Motor auf gewünschter Drezahl Drehen lassen 
			\item Mechanischer Aufbau überprüfen
			\item Test mit Bällen wiederholen 
		\end{enumerate} \\ 
		\hline Ergebnis & Funktion IO/ NIO \\ 
		\hline Verbesserungen & Konstruktive Anpassungen, Drezahl des Motors anpassen, Gegendruckplatte konstruktiv anpassen \\ 
		\hline 
	\end{tabular}
\end{table}

\subsubsection{MB104 Stabilität Aufbau}
\begin{table}[h!]
	\renewcommand{\arraystretch}{1.5}
	\begin{tabular}{|r|p{14cm}|}
		\hline Beschreibung & Der Aufbau muss bei maximaler Drezahl aller Motoren stabil betrieben werden können. \\ 
		\hline Vorbedingungen & Funktionskontrolle der einzelnen Motoren gewährleistet. \\ 
		\hline Vorgehen & 
		\begin{enumerate}
			\item BLDC- Motor auf maximale Drezahl bringen
			\item 2 min drehen lassen 
			\item Schrittmotor mit max. Geschwindigkeit über den ganzen Schwenkbereich drehen lassen  
			\item DC- Motor einschalten
			\item Kontrolle des mechanischen Aufbaus
		\end{enumerate} \\ 
		\hline Ergebnis & Keine Schäden am Aufbau erkennbar/ Schäden am Aufbau erkennbar \\ 
		\hline Verbesserungen & Stabilität anpassen, Drehzahlbereich der Motoren einschränken \\ 
		\hline 
	\end{tabular}
\end{table}
\newpage

\subsubsection{MB105 Funktionskontrolle Treffgenauigkeit }
\begin{table}[h!]
	\renewcommand{\arraystretch}{1.5}
	\begin{tabular}{|r|p{14cm}|}
		\hline Beschreibung & Die Bälle müssen in der gewünschten Zeit die gewünschte Treffgenauigkeit erreichen   \\ 
		\hline Vorbedingungen & Einzelfunktionskontrollen aller Motoren erfolgreich, Stabilität des Aufbaus erreicht \\ 
		\hline Vorgehen & 
		\begin{enumerate}
			\item BLDC- Motor auf gewünschter Drehzahl Drehen lassen 
			\item Ballnachschub starten 
		\end{enumerate} \\ 
		\hline Ergebnis & Funktion IO/ NIO \\ 
		\hline Verbesserungen & Konstruktive Anpassungen, Drezahl des Motors anpassen, Gegendruckplatte konstruktiv anpassen \\ 
		\hline 
	\end{tabular}
\end{table}

\clearpage

\subsection{Testplan-Elektrotechnik}
\subsubsection{Kommunikation PC -- Shell auf Freedomboard kann von PC bedient werden}
\begin{table}[h!]
	\renewcommand{\arraystretch}{1.5}
	\begin{tabular}{|r|p{14cm}|}
		\hline Beschreibung	&
			Das Freedomboard kann per USB/UART-Verbindung bedient werden. \\ 
		\hline Vorbedingungen	&
			Toolchain, Terminalprogramm und Python installiert \\ 
		\hline Testdaten	& - \\ 
		\hline Vorgehen		& 
		\begin{enumerate}
			\item Freedomboard verbinden
			\item Verbindung aufbauen mit Freedomboard
			\item Terminalverbindung eröffnen
			\item Reset des Freedomboards durchführen per Reset-Schalter
			\item Ausgabe begutachten und auf Bereitschaft warten
			\item Terminalverbindung schliessen
			\item Testskript durchführen
			\item Freedomboard trennen von PC 
		\end{enumerate} \\ 
		\hline Ergebnis 	&
			Das Freedomboard zeigt auf der Terminalverbindung nach dem
			Reset die Befehlsliste an. Das Testskript beendet erfolgreich.\\ 
		\hline 
	\end{tabular}
\end{table}

\newpage
\subsubsection{Kommunikation RaspberryPi -- Freedomboard von RaspberryPi bedienbar}
\begin{table}[h!]
	\renewcommand{\arraystretch}{1.5}
	\begin{tabular}{|r|p{14cm}|}
		\hline Beschreibung	&
			Das Freedomboard kann per USB/UART-Verbindung bedient werden. \\ 
		\hline Vorbedingungen	& Python installiert \\ 
		\hline Testdaten	& - \\ 
		\hline Vorgehen		& 
		\begin{enumerate}
			\item Freedomboard verbinden per USB Kabel
			\item Testskript durchführen
			\item Freedomboard trennen
		\end{enumerate} \\ 
		\hline Ergebnis 	&
			Das Testskript beendet erfolgreich.\\ 
		\hline 
	\end{tabular}
\end{table}

\newpage
\subsubsection{Ballabwurf -- BLDC Motor kann per Freedomboard bedient werden}
\begin{table}[h!]
	\renewcommand{\arraystretch}{1.5}
	\begin{tabular}{|r|p{14cm}|}
		\hline Beschreibung	& Der BLDC Motor lässt sich via USB/UART ansteuern. \\ 
		\hline Vorbedingungen	& Shell ist auf dem Freedomboard implementiert. \\ 
		\hline Testdaten	& - \\ 
		\hline Vorgehen		& 
		\begin{enumerate}
			\item Freedomboard verbinen
			\item Verbindung aufbauen mit Freedomboard
			\item Motorspeisung einschalten
			\item Motor einschalten
			\item Geschwindigkeit einstellen in 10\% Schritten
			\item Geschwindigkeit auf 0 stellen
			\item Motor ausschalten
			\item Motorspeisung ausschalten 
		\end{enumerate} \\ 
		\hline Ergebnis 	&
			Der Motor verändert die Geschwindigkeit ensprechend
			den Einstellungen.\\ 
		\hline 
	\end{tabular}
\end{table}

\newpage
\subsubsection{Positionsschalter -- Fixpositionen können per Software erkannt werden}
\begin{table}[h!]
	\renewcommand{\arraystretch}{1.5}
	\begin{tabular}{|r|p{14cm}|}
		\hline Beschreibung	&
			Software kann auf erreichen von Fixpositionen reagieren. \\ 
		\hline Vorbedingungen	&
			Shell ist auf dem Freedomboard implementiert.
			Positionsschalter implementiert. \\ 
		\hline Testdaten	& - \\ 
		\hline Vorgehen		& 
		\begin{enumerate}
			\item Freedomboard verbinden
			\item Verbindung aufbauen zum Freedomboard
			\item Positionserkennung triggern
			\item LED auf Freedomboard betachten
			\item Positionstriggerung quittieren
			\item Freedomboard trennen
		\end{enumerate} \\ 
		\hline Ergebnis 	&
			Die LED auf dem Freedomboard reagiert auf die
			Positionstriggerung und kann per Shell quittiert werden.\\ 
		\hline 
	\end{tabular}
\end{table}

\newpage
\subsubsection{Ballnachschub -- DC Motor kann per Freedomboard bedient werden}
\begin{table}[h!]
	\renewcommand{\arraystretch}{1.5}
	\begin{tabular}{|r|p{14cm}|}
		\hline Beschreibung	& Der DC Motor lässt sich via USB/UART ansteuern. \\ 
		\hline Vorbedingungen	&
			Shell ist auf dem Freedomboard implementiert.
			DC-Treiberstufe implementiert. Positionsschalter implementiert. \\ 
		\hline Testdaten	& - \\ 
		\hline Vorgehen		& 
		\begin{enumerate}
			\item Freedomboard verbinden
			\item Verbindung aufbauen mit Freedomboard
			\item Motor in Mitte der Wegstrecke platzieren 
			\item Verbindungsaufbau PC-Freedomboard
			\item Motorspeisung einschalten
			\item Bewegungsrichtung auf aufwärts einstellen
			\item Motor einschalten
			\item Motor einige Centimeter fahren lassen
			\item Motor ausschalten
			\item Bewegungsrichtung umstellen auf abwärts
			\item Motor einschalten
			\item Motor einige Centimeter fahren lassen
			\item Motor ausschalten
			\item Bewegungsrichtung umstellen auf aufwärts
			\item Motor einschalten
			\item Warten bis Endschalter auslöst
			\item (Bewegungsrichtung umstellen)
			\item Motor einschalten
			\item Warten bis Endschalter auslöst
			\item Motorspeisung ausschalten
			\item Freedomboard trennen
		\end{enumerate} \\ 
		\hline Ergebnis 	&
			Der Motor bewegt sich entsprechend der eingestellten
			Bewegungsrichtung und lässt sich ein- und ausschalten.
			Die Enschalter stoppen die Bewegung des Motors. \\ 
		\hline 
	\end{tabular}
\end{table}

\newpage
\subsubsection{Turmausrichtung -- Schrittmotor kann per Freedomboard bedient werden}
\begin{table}[h!]
	\renewcommand{\arraystretch}{1.5}
	\begin{tabular}{|r|p{14cm}|}
		\hline Beschreibung	&
			Schrittmotor kann per Freedomboard bedient werden. \\ 
		\hline Vorbedingungen	&
			Shell ist auf dem Freedomboard implementiert.
			Positionsschalter implementiert. \\ 
		\hline Testdaten	& - \\ 
		\hline Vorgehen		& 
		\begin{enumerate}
			\item Freedomboard verbinden
			\item Verbindung aufbauen zum Freedomboard
			\item Positionserkennung triggern
			\item LED auf Freedomboard betachten
			\item Positionstriggerung quittieren
			\item Motorspeisung einschalten
			\item Motor in verschiedenen Schrittweiten und Richtungen bewegen
			\item Motor ausschalten
			\item Freedomboard trennen
		\end{enumerate} \\ 
		\hline Ergebnis 	&
			Die LED auf dem Freedomboard reagiert auf die
			Positionstriggerung und kann per Shell quittiert werden.
			Der Motor reagiert entsprechend auf die Ansteuerung. \\ 
		\hline 
	\end{tabular}
\end{table}



\newpage
\input{content/testplan-it}
