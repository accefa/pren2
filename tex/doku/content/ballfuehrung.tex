\subsubsection{Anpressvorrichtung}

\paragraph{Komponentenbeschrieb}
Damit mit verschiedenen Tennisballdurchmessern zuverlässig geschossen werden kann, ist der Anpressvorrichtung mit einer Federkonstruktion erstellt worden. So ist es möglich, den Anpressdruck der Tennisbälle auf das Drehrad konstant zu halten. Da es sich beim Anpressdruck bei unserer Maschine um einen der wichtigsten Parameter handelt, musste dieser Komponente besondere Beachtung geschenkt werden. 

\paragraph{Entwicklungsprozess}
Die Anpressvorrichtung wurde während dem PREN 2 einige Male umkonstruiert und angepasst. In einem ersten Entwurf war die Position des Abwurfes falsch gewählt. Dies führte dazu, dass die Tennisbälle die Maschine nicht wie gewünscht verliessen. Zudem gingen wir davon aus, dass es genügt, die Bälle mit einem elastischen Kunststoff ans Drehrad zu pressen. Dies Funktionierte leider nie zufriedenstellend. Darum wurde eine Konstruktion mit Federn erstellt. Die Wahl der passenden Federn erforderte ebenfalls einige Versuchsdurchführungen. Bei zu geringem Anpressdruck ist die Wurfdistanz zu gering. Analog dazu ist sie zu gross bei zu starke Anpressdruck. Die Konstruktion mit den Federn hat leider auch einen Nachteil. Da die Federkraft, und dadurch auch der Anpressdruck, mit steigendem Federweg zunehmen, variiert die Wurfdistanz trotzdem. Dies passend auszulegen war recht zeitintensiv. Ein weiteres Problem liegt darin, dass wenn die Bälle nicht genau in der Mitte der Anpressvorrichtung der Abwurfposition zugeführt werden können, die Möglichkeit besteht dass diese seitlich abgelenkt werden. Im Nachhinein wäre es wohl einfacher, die Bälle so zu verpacken, dass diese denselben Durchmesser haben und der Anpressdruck fix eingestellt werden kann.