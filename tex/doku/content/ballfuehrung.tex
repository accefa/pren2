\subsubsection{Anpressvorrichtung}
\begin{figure}[h!]
	\centering
	\includegraphics[width=\linewidth]{../../fig/Render-Anpressvorrichtungx}
	\caption{Anpressvorrichtung}
	\label{fig:Anpressvorrichtung}
\end{figure}

\paragraph{Komponentenbeschrieb\\}
Damit mit verschiedenen Tennisballdurchmessern zuverlässig geschossen werden kann, ist die Anpressvorrichtung mit einer Federkonstruktion ausgestattet. Der Anpressdruck der Tennisbälle auf das Drehrad wird so konstant gehalten. Da die gewählte Bauform sensibel auf Änderungen des Anpressdruckes reagiert, musste dieser Komponente besondere Beachtung geschenkt werden. 

\paragraph{Entwicklungsprozess\\}
Die Anpressvorrichtung wurde während dem PREN 2 einige Male umkonstruiert und angepasst. In einem ersten Entwurf war die Position des Abwurfes falsch gewählt. Dies führte dazu, dass die Tennisbälle die Maschine nicht wie gewünscht verliessen. Ein anpressen der Bälle an das Drehrad mit einem elastischen Kunststoff wurde zudem als genügend gut erachtet. Da sich dies in der Ausführung allerdings als nicht zufriedenstellend herausstellte, wurde eine Konstruktion mit Federn erstellt. Um eine funktionierende Befestigung für die Federn zu finden, wurden ebenfalls einige Konstruktionen realisiert. Bei zu geringem Anpressdruck ist die Wurfdistanz zu gering. Analog dazu ist sie zu gross bei zu starkem Anpressdruck. Eine Schwäche zeigt sich jedoch auch bei der gefederten Konstruktion. Da die Federkraft und dadurch auch der Anpressdruck mit steigendem Federweg zunehmen, kann die Wurfdistanz nach wie vor variieren. Dies passend auszulegen war recht zeitintensiv.