\subsubsection{Fotoshoot UI}

Das Fotoshoot UI hat mehrere Funktionen. Es dient um den Ballwurf-Prozess zu starten, die Korberkennung zu kalibrieren und die Motoren anzusteuern.

\paragraph{Starten des Ballwurfs (Komponente Sequence)}
Der gesamte Prozess kann mittels Start-Knopf gestartet werden. Die Benutzeroberfläche wird eingefroren und die Stoppuhr gestartet. Sobald der Prozess beendet wurde, wird die vergangene Zeit angezeigt.

\begin{figure}[h!]
\centering
\includegraphics[width=0.6\linewidth]{../../fig/fotoshoot-ui/fotoshoot-ui-pi}
\caption{Fotoshoot-UI Prozess starten}
\label{fig:fotoshoot-ui-pi}
\end{figure}

\paragraph{Kalibrierung der Bildverarbeitung (Komponente Image)}
In diesem Bereich können die Parameter für die Bild-Erkennung definiert werden. Die Bedeutung der Parameter sowie die gültigen Werte sind in der Tabelle \ref{tab:parameter-bilderkennung} detailliert beschrieben. Im unteren Abschnitt dieses Tabs kann das Bild angezeigt werden. Auf dem Bild ist der erkannte Korb mit einem Fadenkreuz markiert.

\begin{figure}[h!]
	\centering
	\includegraphics[width=0.6\linewidth]{../../fig/fotoshoot-ui/fotoshoot-ui-korb-erkennung}
	\caption{Fotoshoot-UI Bilderkennung}
	\label{fig:fotoshoot-ui-korb-erkennung}
\end{figure}

\paragraph{Ansteuerung der Motoren (Komponente Drive)}
Die einzelnen Motoren können über die Benutzeroberfläche angesteuert werden. Dies vereinfacht das Testen der einzelnen Komponenten massiv und macht es auch für alle zugänglich. Es braucht nun kein spezielles IT-Know-How mehr um die einzelnen Komponenten zu starten.

\begin{figure}[h!]
	\centering
	\includegraphics[width=0.6\linewidth]{../../fig/fotoshoot-ui/fotoshoot-ui-motoren}
	\caption{Fotoshoot-UI Ansteuerung der Motoren}
	\label{fig:fotoshoot-ui-motoren}
\end{figure}

\paragraph{Log (Komponente Logger)}
Während des Ballwurfs werden Log-Einträge erstellt. Die Logger-Komponente kann diese abfragen und bringt diese auf dem Tab Log zur Darstellung.

\begin{figure}[h!]
	\centering
	\includegraphics[width=0.6\linewidth]{../../fig/fotoshoot-ui/fotoshoot-ui-log}
	\caption{Fotoshoot-UI Log Einträge}
	\label{fig:fotoshoot-ui-log}
\end{figure}

\newpage

\paragraph{Technische Umsetzung}
Das Fotoshoot-UI wurde in Java und mit der heute aktuellen UI-Technologie Java-FX implementiert. Zudem sind zentral zwei 3rd-Parties Bibliotheken, welches den Bau dieser Applikation massiv vereinfachen soll.

\begin{description}
	\item[Google Guice:] Dies ist ein Depedency Injection Framework. Die Controller können nun ihre zusätzlichen Abhängigkeiten über die Injektion erhalten. Der Objekt-Baum muss nicht mehr mühsam von Hand erzeugt werden.
	\item[Eventbus:] Der Eventbus ist Teil von Google Guava. Dieser vereinfacht die Implementation des Observer-Patterns. Jeder kann auf den Eventbus Events senden und sich mit lediglich der Annotation \texttt{@Subscribe} für Events registrieren.
\end{description}
\begin{figure}[h!]
	\centering
	\includegraphics[width=0.4\linewidth]{../../fig/fotoshoot-ui/fotoshoot-ui-paketdiagramm}
	\caption{Paketdiagramm Fotoshoot-UI}
	\label{fig:fotoshoot-ui-paketdiagramm}
\end{figure}
Auf der Abbildung \ref{fig:fotoshoot-ui-paketdiagramm} ist zu erkennen wie die Applikation aufgebaut ist. Nachfolgend werden die einzelnen Pakete erklärt:
		
\begin{description}
	\item[Controller:] Die Controller beinhalten die FXML-Dateien und die konkreten Controller Klassen. Sie sind für die Benutzer-Aktionen, Service-Calls, sowie die Beziehung zwischen Modell und UI zuständig.
	\item[Modell:] Dies sind die Modelle fürs GUI. Diese sind fürs GUI zugeschnitten. Die Modelle beinhalten die Properties, welche über die Controller in die Views gebunden werden.
	\item[Services:] Diese sind zuständig für die Daten. Sowohl Daten-Abfrage und auch Manipulation. Sie stellen die Schnittstelle zum Business-Layer dar.
	\item[Events:] Das sind POJOs, welche von den Controller verwendet werden. Diese werden für die Kommunikation der Controller untereinander über den Eventbus benötigt.
\end{description}
Jeder Controller kann der Ursprung für eine Benutzeraktion sein, welcher Einfluss auf irgendwelche Views hat. Diesen Views liegen auch Controller zugrunde. Damit nicht jeder Controller jeden anderen Controller kennen muss, verwenden wir hier den Eventbus von Google. Jeder Controller wird automatisch registriert und kann über den Eventbus Events senden und sich für jedes Event auch wieder registrieren.
