\subsection{Testplan-Elektrotechnik}
\subsubsection{Kommunikation PC -- Shell auf Freedomboard kann von PC bedient werden}
\begin{table}[h!]
	\renewcommand{\arraystretch}{1.5}
	\begin{tabular}{|r|p{14cm}|}
		\hline Beschreibung	&
			Das Freedomboard kann per USB/UART-Verbindung bedient werden. \\ 
		\hline Vorbedingungen	&
			Toolchain, Terminalprogramm und Python installiert \\ 
		\hline Testdaten	& - \\ 
		\hline Vorgehen		& 
		\begin{enumerate}
			\item Freedomboard verbinden
			\item Verbindung aufbauen mit Freedomboard
			\item Terminalverbindung eröffnen
			\item Reset des Freedomboards durchführen per Reset-Schalter
			\item Ausgabe begutachten und auf Bereitschaft warten
			\item Terminalverbindung schliessen
			\item Testskript durchführen
			\item Freedomboard trennen von PC 
		\end{enumerate} \\ 
		\hline Ergebnis 	&
			Das Freedomboard zeigt auf der Terminalverbindung nach dem
			Reset die Befehlsliste an. Das Testskript beendet erfolgreich.\\ 
		\hline 
	\end{tabular}
\end{table}

\newpage
\subsubsection{Kommunikation RaspberryPi -- Freedomboard von RaspberryPi bedienbar}
\begin{table}[h!]
	\renewcommand{\arraystretch}{1.5}
	\begin{tabular}{|r|p{14cm}|}
		\hline Beschreibung	&
			Das Freedomboard kann per USB/UART-Verbindung bedient werden. \\ 
		\hline Vorbedingungen	& Python installiert \\ 
		\hline Testdaten	& - \\ 
		\hline Vorgehen		& 
		\begin{enumerate}
			\item Freedomboard verbinden per USB Kabel
			\item Testskript durchführen
			\item Freedomboard trennen
		\end{enumerate} \\ 
		\hline Ergebnis 	&
			Das Testskript beendet erfolgreich.\\ 
		\hline 
	\end{tabular}
\end{table}

\newpage
\subsubsection{Ballabwurf -- BLDC Motor kann per Freedomboard bedient werden}
\begin{table}[h!]
	\renewcommand{\arraystretch}{1.5}
	\begin{tabular}{|r|p{14cm}|}
		\hline Beschreibung	& Der BLDC Motor lässt sich via USB/UART ansteuern. \\ 
		\hline Vorbedingungen	& Shell ist auf dem Freedomboard implementiert. \\ 
		\hline Testdaten	& - \\ 
		\hline Vorgehen		& 
		\begin{enumerate}
			\item Freedomboard verbinen
			\item Verbindung aufbauen mit Freedomboard
			\item Motorspeisung einschalten
			\item Motor einschalten
			\item Geschwindigkeit einstellen in 10\% Schritten
			\item Geschwindigkeit auf 0 stellen
			\item Motor ausschalten
			\item Motorspeisung ausschalten 
		\end{enumerate} \\ 
		\hline Ergebnis 	&
			Der Motor verändert die Geschwindigkeit ensprechend
			den Einstellungen.\\ 
		\hline 
	\end{tabular}
\end{table}

\newpage
\subsubsection{Positionsschalter -- Fixpositionen können per Software erkannt werden}
\begin{table}[h!]
	\renewcommand{\arraystretch}{1.5}
	\begin{tabular}{|r|p{14cm}|}
		\hline Beschreibung	&
			Software kann auf erreichen von Fixpositionen reagieren. \\ 
		\hline Vorbedingungen	&
			Shell ist auf dem Freedomboard implementiert.
			Positionsschalter implementiert. \\ 
		\hline Testdaten	& - \\ 
		\hline Vorgehen		& 
		\begin{enumerate}
			\item Freedomboard verbinden
			\item Verbindung aufbauen zum Freedomboard
			\item Positionserkennung triggern
			\item LED auf Freedomboard betachten
			\item Positionstriggerung quittieren
			\item Freedomboard trennen
		\end{enumerate} \\ 
		\hline Ergebnis 	&
			Die LED auf dem Freedomboard reagiert auf die
			Positionstriggerung und kann per Shell quittiert werden.\\ 
		\hline 
	\end{tabular}
\end{table}

\newpage
\subsubsection{Ballnachschub -- DC Motor kann per Freedomboard bedient werden}
\begin{table}[h!]
	\renewcommand{\arraystretch}{1.5}
	\begin{tabular}{|r|p{14cm}|}
		\hline Beschreibung	& Der DC Motor lässt sich via USB/UART ansteuern. \\ 
		\hline Vorbedingungen	&
			Shell ist auf dem Freedomboard implementiert.
			DC-Treiberstufe implementiert. Positionsschalter implementiert. \\ 
		\hline Testdaten	& - \\ 
		\hline Vorgehen		& 
		\begin{enumerate}
			\item Freedomboard verbinden
			\item Verbindung aufbauen mit Freedomboard
			\item Motor in Mitte der Wegstrecke platzieren 
			\item Verbindungsaufbau PC-Freedomboard
			\item Motorspeisung einschalten
			\item Bewegungsrichtung auf aufwärts einstellen
			\item Motor einschalten
			\item Motor einige Centimeter fahren lassen
			\item Motor ausschalten
			\item Bewegungsrichtung umstellen auf abwärts
			\item Motor einschalten
			\item Motor einige Centimeter fahren lassen
			\item Motor ausschalten
			\item Bewegungsrichtung umstellen auf aufwärts
			\item Motor einschalten
			\item Warten bis Endschalter auslöst
			\item (Bewegungsrichtung umstellen)
			\item Motor einschalten
			\item Warten bis Endschalter auslöst
			\item Motorspeisung ausschalten
			\item Freedomboard trennen
		\end{enumerate} \\ 
		\hline Ergebnis 	&
			Der Motor bewegt sich entsprechend der eingestellten
			Bewegungsrichtung und lässt sich ein- und ausschalten.
			Die Enschalter stoppen die Bewegung des Motors. \\ 
		\hline 
	\end{tabular}
\end{table}

\newpage
\subsubsection{Turmausrichtung -- Schrittmotor kann per Freedomboard bedient werden}
\begin{table}[h!]
	\renewcommand{\arraystretch}{1.5}
	\begin{tabular}{|r|p{14cm}|}
		\hline Beschreibung	&
			Schrittmotor kann per Freedomboard bedient werden. \\ 
		\hline Vorbedingungen	&
			Shell ist auf dem Freedomboard implementiert.
			Positionsschalter implementiert. \\ 
		\hline Testdaten	& - \\ 
		\hline Vorgehen		& 
		\begin{enumerate}
			\item Freedomboard verbinden
			\item Verbindung aufbauen zum Freedomboard
			\item Positionserkennung triggern
			\item LED auf Freedomboard betachten
			\item Positionstriggerung quittieren
			\item Motorspeisung einschalten
			\item Motor in verschiedenen Schrittweiten und Richtungen bewegen
			\item Motor ausschalten
			\item Freedomboard trennen
		\end{enumerate} \\ 
		\hline Ergebnis 	&
			Die LED auf dem Freedomboard reagiert auf die
			Positionstriggerung und kann per Shell quittiert werden.
			Der Motor reagiert entsprechend auf die Ansteuerung. \\ 
		\hline 
	\end{tabular}
\end{table}

