\section{Schnittstellen}

\section{REST von Webserver}

Die Schnittstelle von der externen Steuereinheit zum Raspberry Pi wird über eine REST\footnote{Representational State Transfer}-Schnittstelle realisiert. Es wurden folgende Ressourcen definiert:
\begin{itemize}
	\item camera
	\item image
	\item start
\end{itemize}
Jede Ressource lässt sich mit HTML-Methoden (GET, PUT, POST usw.) abrufen oder verändern. Die Schnittstelle lässt sich ohne Authentifizierung nutzen und arbeiten mit dem JSON-Datenformat. Nachfolgend werden die einzelnen Ressourcen näher beschrieben.

\subsection{GET camera}

Diese Anfrage liefert die aktuellen Einstellungen der Bilderkennung zurück.

\subsubsection{Parameter}

\begin{tabular}{l p{16cm}}
	\textbf{config} & Die Konfiguration der Bilderkennung \\
	\textbf{roi} & Ist der Bereich in dem der Korb erkennbar ist (Region of Interest). Er ist definiert durch die Koordinaten der linken oberen Ecke (x,y) und einer Fläche (height, width). \\
	\textbf{contrast} & Ein Wert zwischen -100 und 100 für den Kontrast der Kamera. \\
	\textbf{greyscale} & \texttt{true} wenn ein Graustufen-Bild erzeugt werden soll. \\
	\textbf{quality} & Ein Wert zwischen 0 und 100 für die Qualität des JPEG-Bildes. \\
	\textbf{line} & Die Pixellinie welche analysiert werden soll um den Korb zu detektieren. \\
	\textbf{height} & Auf welcher Höhe soll die Pixellinie analysiert werden. Darf max. so hoch sein wie die ROI. \\
	\textbf{area} & Bei der Detektierung wird nicht nur eine Pixellinie untersucht. Mit \texttt{area} kann ein Band definiert werden in dem die Linien analysiert werden. \\
	\textbf{image} & Der URL zum Bild der Kamera. Bei jedem Aufruf wird ein neues Bild erstellt.
\end{tabular}

\subsubsection{Beispiel Request}

\texttt{GET} \\
\texttt{http://<raspberrypi-ip>/camera}

\subsubsection{Beispiel Result}

\begin{lstlisting}[caption=GET camera Result, label=lst:camera, tabsize=2]
{
	"config": {
		"roi": {
			"x": 100,
			"y": 200,
			"height": 300,
			"width": 400
		},
		"contrast": 50,
		"greyscale": true,
		"quality": 100,
		"line": {
			"height": 300,
			"area": 5
		}
	},
	"image": "http://<raspberrypi-ip>/image"
}
\end{lstlisting}

\subsection{PUT camera}

Diese Anfrage verändert die Einstellungen der Bilderkennung.

\subsubsection{Beispiel Request}

\texttt{PUT} \\
\texttt{http://<raspberrypi-ip>/camera} \\
Body siehe Listing \ref{lst:camera}

\subsubsection{Beispiel Result}

\texttt{HTTP/1.1 200 OK} \\
\texttt{Content-Type: text/json}

\subsection{GET image}

Diese Anfrage liefert das Bild der Kamera zurück. Bei jeder Anfrage wird ein Bild erstellt.

\subsubsection{Beispiel Request}

\texttt{GET} \\
\texttt{http://<raspberrypi-ip>/image}

\subsubsection{Beispiel Result}

Bild im JPEG-Format

\subsection{GET start}

Diese Anfrage liefert den Status des Vorgangs zurück. Wenn \texttt{start} auf \texttt{true} gesetzt ist läuft der Vorgang.

\subsubsection{Beispiel Request}

\texttt{GET} \\
\texttt{http://<raspberrypi-ip>/start}

\subsubsection{Beispiel Result}

\begin{lstlisting}[caption=GET start Result, tabsize=2]
{
	"start": true
}
\end{lstlisting}

\subsection{PUT start}

Diese Anfrage startet den Vorgang und übergibt die Callback-Adresse vom Steuergerät, damit der Stopp-Befehl zurück gesendet werden kann.

\subsubsection{Parameter}

\begin{tabular}{l p{16cm}}
	\textbf{start} & Bei \texttt{true} wird der Vorgang gestartet. Der Vorgang wird nicht unterbrochen falls \texttt{start} auf \texttt{false} gesetzt wird. Ist der Vorgang beendet wird \texttt{start} auf \texttt{false} gesetzt \\
	\textbf{url} & Die IP-Adresse des Steuergerätes. Diese wird benötigt um den Stopp-Befehl zurückzusenden.
\end{tabular}

\subsubsection{Beispiel Request}

\texttt{PUT} \\
\texttt{http://<raspberrypi-ip>/start}

\begin{lstlisting}[caption=PUT start Request, tabsize=2]
{
	"start": true,
	"url": "http://<steuergerät-ip>"
}
\end{lstlisting}

\subsubsection{Beispiel Result}

\begin{lstlisting}[caption=PUT start Result, tabsize=2]
{
	"status": true
}
\end{lstlisting}