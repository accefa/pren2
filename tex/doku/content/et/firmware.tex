\subsubsection{Firmware}
Die Firmware des Freedomboard ist in C geschrieben und mit der
Entwicklungsumgebung Kinetis Design Studio des Herstellers
Freescale erstellt. Die Firmware beinhaltet die Implementierung
des Echtzeitbetriebssystems FreeRTOS und die darauf als Task
laufende Shell. Nebst diesem Task gibt es einen weiteren Task,
welcher den Duty-Cycle des PWM-Signal für den BLDC-Motor stellt.
Diese Update-Funktion ist als Task implementiert, da die
Möglichkeit offen ist eine Regelung einzusetzen die den Stellwert
unabhängig von der Shell setzt. Die übrigen Komponenten der
Firmware, wie die Bedienung des DC-Motors und des Schrittmotors
sind als einfache ausführbare Teilprogramme implementiert, welche
über die Shell bedient werden können. Die Shell selbst ist über
die UART-Schnittstelle verfügbar und kann dank der
USB-Seriell-Wandlung auch direkt an einen beliebigen PC verbunden
und bedient werden.


