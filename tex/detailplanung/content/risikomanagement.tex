\section{Risikomanagement}

Die Projektrisiken wurden evaluiert und in Tabelle \ref{tab:risikomanagement} dargestellt. Die einzelnen Risiken wurde auf ihre Wahrscheinlichkeit und Auswirkung bewertet. Je nach Bewertung wurde eine Farbe gemäss Tabelle \ref{tab:risikoreferenz} zugewiesen. Zusätzlich wurde je eine Massnahme zur Prävention des Risiko und eine Massnahme zur Eingrenzung der Auswirkung beschrieben.

\begin{table}[h!]
	\centering
	\begin{tabular}{r || c c c c}
		häufig 		
		& \cellcolor{red} 
		& \cellcolor{red}
		& \cellcolor{red}
		& \cellcolor{red} \\
		wahrscheinlich		
		& \cellcolor{yellow} 
		& \cellcolor{yellow} 
		& \cellcolor{red}
		& \cellcolor{red} \\
		gelegentlich		
		& \cellcolor{yellow}
		& \cellcolor{yellow}
		& \cellcolor{yellow}
		& \cellcolor{red} \\
		vorstellbar		
		& \cellcolor{green}
		& \cellcolor{yellow}
		& \cellcolor{yellow}
		& \cellcolor{yellow} \\
		unwahrscheinlich	
		& \cellcolor{green}
		& \cellcolor{green}
		& \cellcolor{yellow}
		& \cellcolor{yellow} \\
		unvorstellbar		
		& \cellcolor{green}
		& \cellcolor{green}
		& \cellcolor{green}
		& \cellcolor{green} \\
		\hline
		& unwesentlich & geringfügig & kritisch & katastrophal
	\end{tabular}
	\caption{Risikoreferenz}
	\label{tab:risikoreferenz}
\end{table}

\begin{landscape}
	\begin{table}
		\begin{tabular}{|p{5cm}|c|c|p{9cm}|}
			\hline Risiko & Auswirkung & Wahrscheinlichkeit & Massnahmen \\ 
			
			\hline \rowcolor{yellow} IT, MB Projektmitglied fällt aus & geringfügig & vorstellbar & 
			- Vorzeitig über Abwesenheit informieren \newline
			- Arbeitsstand in Gruppe kommunizieren \\ 
			
			\hline \rowcolor{red} ET Projektmitglied fällt aus & katastrophal & gelegentlich & 
			- Wöchentlicher Gesundheits- und Gemütszustand rapportieren \newline
			- Aufbau Know-How in den ET Arbeiten \\
			
			\hline \rowcolor{yellow} \hline Abweichung vom Terminplan & geringfügig & vorstellbar &
			- Realistische Planung \newline
			- Aufarbeiten in Reservezeit \\ 
			
			\hline \rowcolor{yellow} \hline Fehlende Zuverlässigkeit & kritisch & unwahrscheinlich &
			- Klare Aufgabenverteilung \newline
			- Review durch Teammitglieder \\ 
			
			\hline \rowcolor{yellow} \hline Algorithmus zur Korberkennung nicht robust & kritisch & vorstellbar &
			- Gutes Unit-Testing des Algorithmus \newline
			- Früh implementieren und testen \\
			
			\hline \rowcolor{green} \hline Fehlendes Know-How IT & geringfügig & unwahrscheinlich &
			- RaspyPi früh aufsetzen \newline
			- Know-How in Python aufbauen \\ 
			
			\hline \rowcolor{yellow} \hline Integration der ET Schnittstellen in die IT Umgebung & geringfügig & wahrscheinlich &
			- Schnittstellendefinition definieren  \\
		
			\hline \rowcolor{yellow} \hline Beschleunigung der Bälle (Konstanz und Präzision) & kritisch & vorstellbar &
			- Ballnachschub langsamer \newline
			- Metallband installieren \\ 
			
			\hline \rowcolor{yellow} \hline Stabilität beim Abwurf & kritisch & vorstellbar &
			- Mehr Gewicht montieren oder Seitenstütze \newline
			- Dicke der Wände \\ 
			
			\hline \rowcolor{yellow} \hline Falsche Dimensionierung der Motoren & kritisch & vorstellbar &
			- Motoren neu kaufen. \\
			\hline \rowcolor{yellow} \hline Energieversorgung: Umbau von Servernetzteilen gelingt nicht & kritisch & vorstellbar &
			- Andere Energieversorgung wählen \\ 
			
			\hline \rowcolor{yellow} \hline Schwingungen beim Abwurf & kritisch & vorstellbar &
			- Drehrad auswuchten \newline
			- Drehzahl reduzieren \\
			
			\hline \rowcolor{green} \hline Budget & geringfügig & unwahrscheinlich &
			- Budgetplanung laufend aktualisieren  \\ 
			
			\hline \rowcolor{yellow} \hline Achse BLDC nicht stark genug & katastrophal & vorstellbar &
			- Lagerung auf der gegenüberliegenden Seite des Motors erstellen \newline
			- Trägheitsmoment Wurfrad anpassen \newline
			- Langsameres anfahren \\
			
			 
		\end{tabular}
		\caption{Risikomanagement}
		\label{tab:risikomanagement}
	\end{table} 
\end{landscape}
